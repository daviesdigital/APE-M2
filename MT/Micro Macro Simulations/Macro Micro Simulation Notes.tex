\documentclass[11pt]{report}

%use European style
\usepackage[a4paper,left=2cm,right=2cm,top=2cm,bottom=2cm]{geometry}

%few useful packages ------------------------------------------------------------------
\usepackage{setspace}
\let\Tiny=\tiny %remove annoying warnings
\usepackage[english]{babel}
\usepackage[latin1]{inputenc}
\usepackage{amsmath}
\usepackage{amssymb}
\usepackage{amsthm}
\usepackage{amsfonts}
\usepackage{colortbl}
\usepackage{xcolor}
\usepackage{eurosym}
\usepackage{enumitem}
\usepackage{chngpage}
\usepackage{fancyhdr}
\usepackage{fancyvrb}
\usepackage{float}
\usepackage{framed}
\usepackage{multirow}
\usepackage{graphicx}
\graphicspath{ {./images/} }
\usepackage{geometry}
\usepackage{lipsum}
\usepackage{tabularx}
\usepackage[linktocpage]{hyperref}

%define environment for code
\definecolor{orangepse}{RGB}{240,139,39}
\definecolor{redpse}{RGB}{222,6,61}
\newcommand{\rpse}[1]{\textcolor{redpse}{#1}}
\definecolor{dkgreen}{rgb}{0,0.6,0}
\definecolor{gray}{rgb}{0.5,0.5,0.5}
\definecolor{mauve}{rgb}{0.58,0,0.82}

\usepackage{listings}
\lstset{frame=tblr,
	language=R,
	aboveskip=5mm,
	belowskip=5mm,
	showstringspaces=false,
	columns=flexible,
	basicstyle={\small\ttfamily},
	numbers=none,
	numberstyle=\tiny\color{gray},
	keywordstyle=\color{blue},
	commentstyle=\color{dkgreen},
	stringstyle=\color{mauve},
	breaklines=true,
	breakatwhitespace=true,
	tabsize=3
}
%---------------------------------------------------------------------------------------


% New Commands ----------------------------
\newcommand{\bb}{\bigbreak\noindent}

\makeatletter
\renewcommand\section{\leftskip 0pt\@startsection {section}{1}{\z@}%
	{-3.5ex \@plus -1ex \@minus -.2ex}%
	{2.3ex \@plus.2ex}%
	{\normalfont\Large\bfseries}}

\renewcommand\subsection{\leftskip 4ex\@startsection{subsection}{2}{\z@}%
	{-3.25ex\@plus -1ex \@minus -.2ex}%
	{1.5ex \@plus .2ex}%
	{\normalfont\large\bfseries}}

\renewcommand\subsubsection{\leftskip 14ex\@startsection{subsubsection}{3}{\z@}%
	{-3.25ex\@plus -1ex \@minus -.2ex}%
	{1.5ex \@plus .2ex}%
	{\normalfont\large\bfseries}}
\makeatother

%----------------------------------------------------------

\definecolor{titlepagecolor}{cmyk}{1,.60,0,.40}
\definecolor{namecolor}{cmyk}{1,.50,0,.10} 


\begin{document}
	\setcounter{page}{1}
	\begin{spacing}{1.5}
		
		% Table of Contents ----------------------------------
		\tableofcontents
		\setcounter{secnumdepth}{-2}
		\newpage
		
		% Start of Content ---------------------------------------------
	\chapter{General Notes:} 
		
		First four classes - Microsimulation\\
		Last four classes - Macrosimulation
		\bb
		Practical Evaluation - Python completing code 
		
	\chapter{Lecture 1: Microsimulation in Static Models}
		Where as random control trials and natural experiments measure the impact of a policy \textit{ex post}, micro and macro simulations were created to find the impacts \textit{ex ante}.
		
		\section{Model Complexity}
		
			\subsection{Population Complexity}
			Static model with no temporal element.\\
			You have a database of a population with as many characteristics as you can gather about them. \\
			See how this is effected by a proposed policy.
			
			\subsection{Behavioural Complexity}
			
			\subsection{Temporal (Dynamic) Complexity}
			
			\subsection{Spacial Complexity}
			
		\section{Typology of microsimulation models}
			\subsection{Hypothetical Model}
			Models tested using an synthetic/artificial population of households/individuals. \\
			Used for:
			\begin{itemize}[leftmargin=10ex]
				\item Illustrative purposes
				\item Validation
				\item Cross country comparisons
			\end{itemize}
			
				\subsubsection{Limitations}
				As you can imagine this method has its own issues
				\begin{itemize}[leftmargin=15ex]
					\item Limited heterogeneity
					\item Lack of representativeness
					\item Will often disregard detailed aspects of policy that matters a lot
				\end{itemize}
				
			\subsection{Static Models}
			Models which use some form of micro-data, but no behavioural or temporal conditions.
			This method provides a focus on the complexity of a policy interacted with the
			complexity of population \& ``day after reform" effects
			
			\subsection{Behavioural Models}
			
			\subsection{Dynamic Models}
			
			
		\section{Static Models}
			\subsection{Baseline Data}
			First you must build it:
			\begin{itemize}[leftmargin=10ex]
				\item Using Admin data and Survey Data
			\end{itemize}
			Then you must maintain it:
			\begin{itemize}[leftmargin=10ex]
				\item This brings a lag of a few years often
			\end{itemize}
			
			
			\subsection{Coding Policies}
			
		
		
	\chapter{Behavioural Responses and Dynamic Models}
		In previous lecture we saw models with no behavioural dynamics and no time dimension.
		
		\section{Structural Models}
		
		\section{Reduced Form}
		
		
\chapter{Macro Simulation}

	\section{The Basic New Keynesian Model : Fiscal and Monetary Policies}
	Check Slides
	
	\section{Applied DSGE models: Impact of the Financial Acts during the Covid19 crisis}
	
		\subsection{Extensions to previous Framework}
		Let's look at how to extend the basic New Keynesian model we saw earlier.
		
			\subsubsection{Ricardian Households:}
			In the utility function of the agent we add a variable representing government expenditure. This suggests that individuals utility/welfare is dependent not only on private consumption, but also private consumption.
			
			\subsubsection{Openness of the Economy:}
			We want to represent the important feature of open economies like France. Especially when analysing the Covid period and the effect of energy cost shocks. Oil is bought from a foreign market. 
			
			\subsubsection{Introduction of Taxation:}
			Taxation on wages and consumption.
			
	
	
	
		
	\end{spacing}
\end{document}
