\documentclass[11pt]{report}

%use European style
\usepackage[a4paper,left=2cm,right=2cm,top=2cm,bottom=2cm]{geometry}

%few useful packages ------------------------------------------------------------------
\usepackage{setspace}
\let\Tiny=\tiny %remove annoying warnings
\usepackage[english]{babel}
\usepackage[latin1]{inputenc}
\usepackage{amsmath}
\usepackage{amssymb}
\usepackage{amsthm}
\usepackage{amsfonts}
\usepackage{colortbl}
\usepackage{xcolor}
\usepackage{eurosym}
\usepackage{enumitem}
\usepackage{chngpage}
\usepackage{fancyhdr}
\usepackage{fancyvrb}
\usepackage{float}
\usepackage{framed}
\usepackage{multirow}
\usepackage{graphicx}
\graphicspath{ {./images/} }
\usepackage{geometry}
\usepackage{lipsum}
\usepackage{tabularx}
\usepackage[linktocpage]{hyperref}

%define environment for code
\definecolor{orangepse}{RGB}{240,139,39}
\definecolor{redpse}{RGB}{222,6,61}
\newcommand{\rpse}[1]{\textcolor{redpse}{#1}}
\definecolor{dkgreen}{rgb}{0,0.6,0}
\definecolor{gray}{rgb}{0.5,0.5,0.5}
\definecolor{mauve}{rgb}{0.58,0,0.82}

\usepackage{listings}
\lstset{frame=tblr,
	language=R,
	aboveskip=5mm,
	belowskip=5mm,
	showstringspaces=false,
	columns=flexible,
	basicstyle={\small\ttfamily},
	numbers=none,
	numberstyle=\tiny\color{gray},
	keywordstyle=\color{blue},
	commentstyle=\color{dkgreen},
	stringstyle=\color{mauve},
	breaklines=true,
	breakatwhitespace=true,
	tabsize=3
}
%---------------------------------------------------------------------------------------


% New Commands ----------------------------
\newcommand{\bb}{\bigbreak\noindent}

\makeatletter
\renewcommand\section{\leftskip 0pt\@startsection {section}{1}{\z@}%
	{-3.5ex \@plus -1ex \@minus -.2ex}%
	{2.3ex \@plus.2ex}%
	{\normalfont\Large\bfseries}}

\renewcommand\subsection{\leftskip 4ex\@startsection{subsection}{2}{\z@}%
	{-3.25ex\@plus -1ex \@minus -.2ex}%
	{1.5ex \@plus .2ex}%
	{\normalfont\large\bfseries}}

\renewcommand\subsubsection{\leftskip 14ex\@startsection{subsubsection}{3}{\z@}%
	{-3.25ex\@plus -1ex \@minus -.2ex}%
	{1.5ex \@plus .2ex}%
	{\normalfont\large\bfseries}}
\makeatother

%----------------------------------------------------------

\definecolor{titlepagecolor}{cmyk}{1,.60,0,.40}
\definecolor{namecolor}{cmyk}{1,.50,0,.10} 


\begin{document}
	\setcounter{page}{1}
	\begin{spacing}{1.5}
		
		% Table of Contents ----------------------------------
		\tableofcontents
		\setcounter{secnumdepth}{-2}
		\newpage
		
		% Start of Content ---------------------------------------------
		\chapter{General Notes:} 
		
		\chapter{Lecture 1: Introduction}
		\section{Patterns of FDI}
		
		\subsection{Section 1:}
		
		
	\chapter{Horizontal FDI}
		Should we invest in a factory abroad or should be invest on increasing production at home and exporting?
		
		\section{Proximity vs. Concentration}
		Producing in proximity to customers abroad saves trade costs.
		\bb
		Concentration (at home) and exporting saves additional production fixed costs, and allows enjoying greater economies of scale.
		\bb
		Firms choose between these alternatives by comparing respective profits. 
		\begin{itemize}
			\item The larger trade costs- the more likely to observe firms choosing to serve destination by setting up affiliate.
			\item The larger plant fixed costs (and hence economies of/returns to scale) the more likely to observe firms concentrating production at home and exporting.
		\end{itemize}
		
		
		\section{Theory and Evidence for the Proximity-Concentration Trade-off in the Presence of Heterogenous Firms}
		Exporters (trading firms) are bigger, better, faster... different. \\
		Different firms within an industry may resolve the proximity-concentration trade-off in opposite directions (Ex/FDl).
		
		\subsection{HMY (2004)}
		\subsubsection{Positive Sorting Assumption (Productivity)}
		MNE's are more productive than exporters.\\
		Exporters are more productive than domestic firms.
		
		
		\section{Irarrazabal, Moxnes and Opromolla (2013): The Margins of Multinational Production and the Role of Intrafirm Trade}
		Why do affiliate sales drop with distance (from the head quarters)? In theory, this shouldn't happen as the whole point of an affiliate business is to avoid this. 
		\bb
		Firms opt for affiliates over exporting to avoid the trade barriers that come with exporting over distance.
		
		\bb
		HMY (2004): affiliate sales should increase with distance at the aggregate level. Within firm distance shouldn't affect anything. 
		
		\subsection{Main Takeaways}
		Convincing story: parents transfer inputs to affiliates.
		This transfer is estimated to be an important part costs.
		Distance, trade barriers increase cost/lower efficiency of transfer.
		Great data, very elegant estimation, model, validation exercises.
		Read this paper for inspiration!	
		
		
		\section{Cross-Border M\&A or Greenfield Investment?}
		So far we have assumed an affiliate business is a greenfield investment. However, M\&As (where the buyer gets more than 10 percent of the business) make over 50 percent of FDI.
		\begin{itemize}
			\item Removes fixed costs associated with building plants
		\end{itemize}
		
		\subsection{Differences between two modes of entry}
		On \textbf{buyer side}:\\
		Nocke \& Yeaple (2008): Higher productivity of U.S. parents predicts higher likelihood of green�eld investment, versus M\&A. $\rightarrow$ Suggests sorting across these two modes of entry.
		\bb
		On \textbf{target side}:\\
		Arnold \& Javorcik (2009), Guadalupe, Kuzmina \& Thomas (2012): Higher productivity predicts likelihood of being a target of M\&A. \\
		Recall: theory above predicts weaker firms get purchased.
		
		\subsection{Brown-field investments}
		
	\chapter{Vertical FDI}
	Horizontal FDI motivated by serving demand in Foreign market.\\
	FDI occurs mostly between developed economies, but when less developed economies involved, unidirectional North $\rightarrow$ South.
	\begin{itemize}
		\item Parents do most of R\&D. Affiliates do not do much R\&D.
		\item Affiliates sell mostly to host country, but not all.
	\end{itemize}
	
	\bb
	\textbf{\textit{Vertical foreign direct investment}} occurs when a multinational acquires an operation that either acts as a supplier or distributor.
	
		\section{General Equilibrium Theory of Vertical MNEs, Krugman +}
		
			Positive:
			\begin{itemize}
				\item Volume of MP increases with international factor-price differences.
				\item Cost saving determines the location of production.
				\item In contrast to Horizontal FDI model, where demand is more important for location decisions.
			\end{itemize}
			
			Normative:
			\begin{itemize}
				\item Emergence of MNEs improves global factor allocation.
				\item Mechanism: Factor price equalization (FPE) in a larger set of endowments compared
			
			\end{itemize}
			
			
		
	
	\end{spacing}
\end{document}
