\documentclass[11pt]{report}

%use European style
\usepackage[a4paper,left=2cm,right=2cm,top=2cm,bottom=2cm]{geometry}

%few useful packages ------------------------------------------------------------------
\usepackage{setspace}
\let\Tiny=\tiny %remove annoying warnings
\usepackage[english]{babel}
\usepackage[latin1]{inputenc}
\usepackage{amsmath}
\usepackage{amssymb}
\usepackage{amsthm}
\usepackage{amsfonts}
\usepackage{colortbl}
\usepackage{xcolor}
\usepackage{eurosym}
\usepackage{enumitem}
\usepackage{chngpage}
\usepackage{fancyhdr}
\usepackage{fancyvrb}
\usepackage{float}
\usepackage{framed}
\usepackage{multirow}
\usepackage{graphicx}
\graphicspath{ {./images/} }
\usepackage{geometry}
\usepackage{lipsum}
\usepackage{tabularx}
\usepackage[linktocpage]{hyperref}

%define environment for code
\definecolor{orangepse}{RGB}{240,139,39}
\definecolor{redpse}{RGB}{222,6,61}
\newcommand{\rpse}[1]{\textcolor{redpse}{#1}}
\definecolor{dkgreen}{rgb}{0,0.6,0}
\definecolor{gray}{rgb}{0.5,0.5,0.5}
\definecolor{mauve}{rgb}{0.58,0,0.82}

\usepackage{listings}
\lstset{frame=tblr,
	language=R,
	aboveskip=5mm,
	belowskip=5mm,
	showstringspaces=false,
	columns=flexible,
	basicstyle={\small\ttfamily},
	numbers=none,
	numberstyle=\tiny\color{gray},
	keywordstyle=\color{blue},
	commentstyle=\color{dkgreen},
	stringstyle=\color{mauve},
	breaklines=true,
	breakatwhitespace=true,
	tabsize=3
}
%---------------------------------------------------------------------------------------


% New Commands ----------------------------
\newcommand{\bb}{\bigbreak\noindent}

\makeatletter
\renewcommand\section{\leftskip 0pt\@startsection {section}{1}{\z@}%
	{-3.5ex \@plus -1ex \@minus -.2ex}%
	{2.3ex \@plus.2ex}%
	{\normalfont\Large\bfseries}}

\renewcommand\subsection{\leftskip 4ex\@startsection{subsection}{2}{\z@}%
	{-3.25ex\@plus -1ex \@minus -.2ex}%
	{1.5ex \@plus .2ex}%
	{\normalfont\large\bfseries}}

\renewcommand\subsubsection{\leftskip 14ex\@startsection{subsubsection}{3}{\z@}%
	{-3.25ex\@plus -1ex \@minus -.2ex}%
	{1.5ex \@plus .2ex}%
	{\normalfont\large\bfseries}}
\makeatother

%----------------------------------------------------------

\definecolor{titlepagecolor}{cmyk}{1,.60,0,.40}
\definecolor{namecolor}{cmyk}{1,.50,0,.10} 


\begin{document}
	\setcounter{page}{1}
	\begin{spacing}{1.5}
		
		% Table of Contents ----------------------------------
		\tableofcontents
		\setcounter{secnumdepth}{-2}
		\newpage
		
		% Start of Content ---------------------------------------------
		\chapter{General Notes:} 
		
		\chapter{Lecture 1: Measuring Aging}
		Defining ageing.
		\begin{itemize}
			\item Demographic Process
			\item Change in health
			\item Change in productivity
		\end{itemize}
		
		\section{Life Tables}
		\textbf{EXAM QUESTION EVERY YEAR:} What is the difference between period and cohort life tables
		
		\subsection{Observed data vs fictitious cohort}
		
		\begin{itemize}[leftmargin=10ex]
			\item  Period life table can be observed at each period, but does  not reflect mortality experience of a real cohort
			\item  Cohort life table can be observed once every individual of a cohort has died.
		\end{itemize}
		
		
		
		\section{Old Age Dependency Ratios}
		\[ \dfrac{Number \:\: of \:\: People \:\:Older \:\: than \:\:65}{Number \:\: of \:\: People \:\:between \:\: 15-65} \]
		\[ \dfrac{D_{65+}}{N_{15-64}} \]
		
	\section{Ageing by the Top}
	
		\subsection{From the bottom}
			Ageing driven by fertility decrease : reductions in number of youths or prime aged individuals
			\begin{itemize}[leftmargin=10ex]
				\item Historically decline in infant mortality led to increase in the share of the 60+
			\end{itemize}
			
		\subsection{From the Top}
		More recent realisation that recent ageing process is essentially ageing by the top
		\begin{itemize}[leftmargin=10ex]
			\item Recent gains in life expectancy come from gains at older ages
			\item Variety of experience at international level
		\end{itemize}
		
		
	\chapter{Institutional design of Pensions}
		\section{History Of Retirement}
			\subsection{Family Support}
			\subsection{Charity and Assistant}
			\subsection{Occupational Pensions}
			\subsection{Individual Savings}
			\subsection{Birth of the Welfare State: Bismark}
				\subsubsection{Work Related Accident Assurance}
				\subsubsection{Healthcare Insurance}
				Employers contributed one-third, the workers two-third\\
				``Sickness funds"", managed by workers' representatives
				
				\subsubsection{Old-age and disability insurance (1889)}
				Participation was mandatory (except for civil servants,covered by previous scheme) \\
				All workers concerned (not only industry workers)\\
				Contributory system funded by employee, employers and
				the State\\
				Pension age was set at 70. 
				\bb
				This was not about enjoying life after a career of hard work. This was strictly pragmatic. What do you expect from the Germans?
				
			\subsection{The UK and the Beverage Report}
			Unlike Bismark, the objective of the Beverage report was to lift all British out of poverty, but not to provide high replacement rates.
			\bb
			It was a comprehensive report that factored in all parts of living standards from cradle to grave. (Health system, Education, Housing, etc.)
			\bb
			An attack on the 5 evils:
			\begin{center}
				\textit{Want, Disease, Ignorance, Squalor, and Idleness.}
			\end{center}
			
		
		\section{Rationale for Public Intervention}
		
			\subsection{Market Failures}
				\subsubsection{Capital Markets}
				Large volatility in capital market returns\\
				Inflation could wipe out the value of someone's holdings very quickly in rare events.
				\bb
				Governments began to create index-linked bonds in an attempt to attenuate the effects of those events.
				
				
			\subsection{Myopia}
			People are often short sighted, and so they under-save for retirement
			
			\subsection{Samaritan's dilemma}
			If there is expectation that there will be assistance to the elderly poor (i.e., elderly cannot be left dying) Then some will under-save for retirement, expecting receiving welfare when poor
			\bb
			Governments will the intervene to attenuate the motivation to game the system.
			
			\subsection{Redistribution}
			Both within and between cohorts. (Some cohorts experience different shocks, large systems help smooth the effects of these)\\ 
			General objective to prevent poverty, especially of elderly individuals.
			
			\subsection{Efficiency and Administrative Costs}
			One of the benefits of compulsion is that no money is spent on selling insurance products by the state. But this harms competition and a uniform scheme might not accommodate heterogeneity in
			preferences (if too big).
			
		\section{Pension Design around the World}
		
			\subsection{Bismark vs Beveridge }
			
			\begin{figure}[h!]
				\centering
				\includegraphics[width=0.7\linewidth]{screenshot001}
			\end{figure}
			
			Historically more complex\\
			Beveridge plan was in the form of a social insurance
			\begin{itemize}[leftmargin=10ex]
				\item Funded by National insurance contributions (NICs)
				\item Contributory benefits proportional to years of contribution
			\end{itemize}
			\bb
			Big difference : benefits expressed as absolute amount (not	as share of earnings). Evolution lead to marked difference with earnings-related schemes
		
		
		\section{Types of Pensions}
			\subsection{Public vs mandatory private vs voluntary private}
			\begin{itemize}[leftmargin=10ex]
				\item Mandatory systems can be public or private
				\item Mandate can be found with scheme monopoly or
				competition
				\item Public schemes can be run by the State or Social security
				administrations
			\end{itemize}
		
			\subsection{Funded vs unfunded vs mixed funding}
			Think of funding more as a spectrum with complete funding and unfunded being corner solutions.
			\begin{itemize}[leftmargin=10ex]
				\item  Funded : contributions invested in capital markets
				\item Unfunded or PAYGO : contributions directly used to
				finance current pensions
				\item Mixed funding : PAYGO with some reserves
			\end{itemize}
		
		
\chapter{Pensions and Capital Markets}

	\section{Funded vs Unfunded}
	
		\subsection{Chocolate Economy - Paul A. Samuelson}
		Imagine an economy with no durable goods, just goods like chocolate which you cannot store.
		There are two periods:
		\begin{enumerate}[leftmargin=10ex]
			\item You work
			\item You retire
		\end{enumerate}
		Now lets introduce an unfunded pension scheme, where workers give a fraction of their income to the retirees.
		\bb
		Samuelson claims in his model that Unfunded pension system offers implicit rate of returns equal to the growth of the tax base, approximately $n + g$. 
		\bb
		However, this is contingent on the consistent growth of the population.
		\bb
		He states you can get the same result with \textbf{fiat money.} This suggests that the ownership of durable capital is what defines a Funded system.
		
			\subsubsection{NB. EXAM QUESTION}
			Understand that investing in treasury bonds is feature of an unfunded system. \\
			Funded systems require on the \textit{\textbf{investment and ownership of capital which will produce returns}}
		
		\subsection{Peter Diamond - National Debt in Neoclassical Growth Model}
		Now lets acknowledge the existence of capital. Despite, three decade in which the rate of return for capital is lower than the growth of the economy, unfunded systems are not pareto-efficient.
		\bb
		Loss/gain from unfunded pension is $(r - \gamma)\tau w_t L_t$. We can see that if $r<\gamma$ then you see a loss yet for most of recent history r.
		
		\subsection{Labour Supply Response}
	
	\section{Adequacy of Savings}
	Are people aware of how much money they should save or be saving?
	
		\subsection{Drop in consumption at retirement}
		
		Reasons for expecting lower consumption needs
		\begin{itemize}[leftmargin=10ex]
			\item change in housing size or location
			\item with children gone, lower expenses
			\item substitution of market expenditures to home production
		\end{itemize}
		\bb
		Reasons for expecting higher consumption needs
		\begin{itemize}[leftmargin=10ex]
			\item enjoying leisurely activities
			\item out-of pocket health care costs
			\item long-term care at older ages
		\end{itemize}
		
		\bb
		Empirically however we see that consumption usually drops upon retirement.\\
		Evidence also shows that this is negatively correlated with wealth (with consumption of wealthier dropping less than the poorer)
		
		\subsubsection{Solving the Puzzle}
		While it is true that \textit{\textbf{expenditure}} drops in retirement (due to reduced work related expenditure and an increase in non-market home production like cooking) \textit{\textbf{consumption}} (such as caloric intake) remains very similar! This is an important shift in understanding.
		
		
	
	\section{Impact of Pensions on Savings}
	Do pensions crowd out other investments?
	\bb
	 If so the impact of an unfunded pension system would be hugely detrimental!
	 
	 \bb
	 REVIEW SLIDES FOR LITERATURE.
	 
	\section{Retirement Saving Policies}
	
		\subsection{Tax Incentives}
		There exists tax favoured savings accounts (IRA in US, PERP in France)
		\bb
		Or a defined contribution account (401K)
		\bb
		\begin{itemize}[leftmargin=10ex]
			\item No tax on contribution
			\item Interest is accumulated tax free
			\item Taxes are paid on withdrawal
		\end{itemize}
		(Roth IRAs work a little bit differently)
		
		\bb
		Keep in mind that Tax Incentives have both substitution effect (which suggests you will substitute consumption for saving) and \textbf{\textit{income effects}} (which suggest that saving the save amount has made you richer and so you might choose to save a little less)
		
		
\chapter{Pensions and Labour Markets}
What is retirement?\\
In economics retirements is viewed as withdrawing from the labour force. The claiming of a pension is not necessarily involved, as there is no reason why both have to happen simultaneously (even though they tend to). 
\bb
Yet retirement is ambiguous. Are people retired or unemployed? People may be retired and then choose to re-enter the workforce. People also may not completely withdraw from the labour force, but partially retire by reducing their hours.
\bb
Labour force participation for those 65 and over has dropped significantly in 100 years.


	\section{Modelling Retirement}
		\subsection{A Simple Retirement Model}
			\subsubsection{Income effects}
			Long-term trend in decreasing retirement age, might be explained by income effects.\\
			Being richer, we consume more leisure
			
			\subsubsection{Limits}
			Value of market time independent of age. Value of leisure independent of age. \\
			No reason for bunching of leisure at the end of life.
			
		\subsection{Option Value Model}
		
		\subsection{Dynamic Programming Models}
		
		
	\section{Impact of Pensions on Incentives}
	
		\subsection{Retirement Spikes}
		
		\subsection{Pension Incentives}
		
		\subsection{NB ALWAYS A QUESTION ABOUT THE EARNINGS TEST}
		Following these kinds of studies most countries have removed the earnings test.
		
		\subsection{Changes in Retirement age (Bozio did his PhD on this)}
		Use of Diff-in-Diff when exact date of birth not known.\\
		Ideally RDD designs are used.
		
\chapter{Redistribution and Well-Being}
The objective of a pension system is by definition to redistribute wealth from those who die early to those who die late. They act as an insurance against the risk of living a long time. It would be an insurance if mortality was completely random. 

\section{Redistribution}
		
	\subsection{Poverty \& Differential Life-Expectancy:}
	Unemployed people are significantly more likely to die at a given age than those who are in employment.
	\bb
	Life expectancy also increases as income increases. Furthermore, at the poorest end where you live has an impact.
	\bb
	Deaths from cardiovascular disease contributed most to the gap in life expectancy between income quartiles, with cancer (lung especially) being second.

	\subsection{Elderly Poverty}
	Measured in absolute terms in the US and in relative terms in the EU. Can also use living situations as poverty can lead to multi-generational cohabitation.
	
	\bb
	The causal impact of social security benefits is tough to establish. 
	
	\subsection{Redistribution Across Cohorts}
	
	\subsection{Redistribution Within Cohorts}
		
		\subsubsection{Annual vs Lifetime}
		\begin{itemize}[leftmargin=20ex]
			\item Annual: workers to elderly
			\item Lifetime: largely an empirical question
		\end{itemize}
		
		 
		 \subsubsection{Leibman, 2022: US}
	 
	 
		\subsubsection{Germany}
		
		
	\section{Health \& Well-Being}
		
		\subsection{Impact of Health on Retirement}
		Does health status affect recession decisions ?
		
		\subsection{Impact of Retirement on Health}
		Does retirement affect your health/mortality?
		\begin{itemize}[leftmargin=10ex]
			\item Meta-analysis has shown that there is no or very little impact of retirement age on mortality.
			\item Retirement leads to better self-reported health.
			\item Retirement leads to a decrease of cognitive functions.
		\end{itemize}
		
		\subsubsection{Work Quality}
		There is a link between the quality of ones job and their health in retirement.
		
		
\chapter{Longterm Care for the Elderly}
We use surveys to measure people's abilitiy to look after them selves (APL and iADL).

	\section{Insuring against long-term care risk}
	There is a large and uncertain risk suggests great value to insurance. 35\%-50\% of 65 year-old will use nursing home (in the U.S.) among which 10-20\% more than 5 years.  
	
		\subsection{Why so little private insurance?}
		U.S. 14\% of 60+ had a long-term care insurance policy. Typical policy only covers 2/3 of long-term care cost, with
		a premium of \$4,500 per year.
		
		\begin{itemize}[leftmargin=10ex]
			\item \textbf{Supply side market failures}
			\begin{itemize}
				\item asymmetric information (adverse selection and moral hazard)
				\item imperfect competition
				\item transaction costs
			\end{itemize}
			
			\item \textbf{Limited Demand}
			\begin{itemize}
				\item Imperfect but cheaper substitutes (Medicaid in the US, Informal Care, money from kids)
				\item Limited Rationality
			\end{itemize}
		\end{itemize}
		
		\subsubsection{title}
		
\chapter{Pension Reforms}
Motivated by significant demographic changes (baby boomers retiring, increased life expectancy, lower birthrate), countries are having to reform their pension schemes. These reforms will have significant impacts on their public finances.

	\section{Policy Options}
	\begin{enumerate}
		\item Higher level of funding
		\begin{itemize}
			\item Higher public funding
			\item Higher savings rates
		\end{itemize}
		
		\item Radical reform (privatisation)
		\begin{itemize}
			\item Switch from unfunded to funded
		\end{itemize}
		
		
		\item Parametric reform
		\begin{itemize}
			\item Increase contributions
			\item Reduce benefits
			\item Increase retirement age
		\end{itemize}
		
	\end{enumerate}
	
	
	\section{Increased Funding}
	Benefits of funded systems
	\begin{itemize}
		\item Higher return on funded systems
		\item Lower costs of running pension
		\item Higher savings and higher investments $\rightarrow$ Higher economic growth
	\end{itemize}
	Demographic changes also impact funded systems (higher life expectancy $\rightarrow$ lower annuity). This is particularly problematic for countries with a declining population.
	
		\subsection{Transition from unfunded to funded}
		\begin{itemize}[leftmargin=10ex]
			\item Cut current pension benefits (pisses off retirees)
			\item Ask cohort to pay twice (pisses off workers)
			\item Issue debt to pay for it (pisses off those who cant vote yet)
		\end{itemize}
		Unfunded systems have an implicit debt, and so to transition governments would have to pay that debt.
		\textbf{The implicit debt of Pensions is often multiples of GDP in size.} Pensions account for a large percentage of annual government spending. To do that each year and secure it, you would need a large loan.
		
	\section{Privatisation}
	Only Chile has managed to do it. 
	
	
	\section{Reforming PAYGO}
	
	\end{spacing}
\end{document}
