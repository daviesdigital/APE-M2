\documentclass[11pt]{report}

%use European style
\usepackage[a4paper,left=2cm,right=2cm,top=2cm,bottom=2cm]{geometry}

%few useful packages ------------------------------------------------------------------
\usepackage{setspace}
\let\Tiny=\tiny %remove annoying warnings
\usepackage[english]{babel}
\usepackage[latin1]{inputenc}
\usepackage{amsmath}
\usepackage{amssymb}
\usepackage{amsthm}
\usepackage{amsfonts}
\usepackage{colortbl}
\usepackage{xcolor}
\usepackage{eurosym}
\usepackage{enumitem}
\usepackage{chngpage}
\usepackage{fancyhdr}
\usepackage{fancyvrb}
\usepackage{float}
\usepackage{framed}
\usepackage{multirow}
\usepackage{graphicx}
\graphicspath{ {./images/} }
\usepackage{geometry}
\usepackage{lipsum}
\usepackage{tabularx}
\usepackage[linktocpage]{hyperref}

%define environment for code
\definecolor{orangepse}{RGB}{240,139,39}
\definecolor{redpse}{RGB}{222,6,61}
\newcommand{\rpse}[1]{\textcolor{redpse}{#1}}
\definecolor{dkgreen}{rgb}{0,0.6,0}
\definecolor{gray}{rgb}{0.5,0.5,0.5}
\definecolor{mauve}{rgb}{0.58,0,0.82}

\usepackage{listings}
\lstset{frame=tblr,
	language=R,
	aboveskip=5mm,
	belowskip=5mm,
	showstringspaces=false,
	columns=flexible,
	basicstyle={\small\ttfamily},
	numbers=none,
	numberstyle=\tiny\color{gray},
	keywordstyle=\color{blue},
	commentstyle=\color{dkgreen},
	stringstyle=\color{mauve},
	breaklines=true,
	breakatwhitespace=true,
	tabsize=3
}
%---------------------------------------------------------------------------------------


% New Commands ----------------------------
\newcommand{\bb}{\bigbreak\noindent}

\makeatletter
\renewcommand\section{\leftskip 0pt\@startsection {section}{1}{\z@}%
	{-3.5ex \@plus -1ex \@minus -.2ex}%
	{2.3ex \@plus.2ex}%
	{\normalfont\Large\bfseries}}

\renewcommand\subsection{\leftskip 4ex\@startsection{subsection}{2}{\z@}%
	{-3.25ex\@plus -1ex \@minus -.2ex}%
	{1.5ex \@plus .2ex}%
	{\normalfont\large\bfseries}}

\renewcommand\subsubsection{\leftskip 14ex\@startsection{subsubsection}{3}{\z@}%
	{-3.25ex\@plus -1ex \@minus -.2ex}%
	{1.5ex \@plus .2ex}%
	{\normalfont\large\bfseries}}
\makeatother

%----------------------------------------------------------

\definecolor{titlepagecolor}{cmyk}{1,.60,0,.40}
\definecolor{namecolor}{cmyk}{1,.50,0,.10} 


\begin{document}
	\setcounter{page}{1}
	\begin{spacing}{1.5}
		
		% Table of Contents ----------------------------------
		\tableofcontents
		\setcounter{secnumdepth}{-2}
		\newpage
		
		% Start of Content ---------------------------------------------
	\chapter{General Notes:} 
	The aim of the course is to acquaint students with the central mechanisms through which financial variables (credit flows, credit spreads, asset prices) interact with macroeconomic variables such as consumption, investment and GDP over the business cycle and at times of financial crises. 
	\bb
	30\% Referee Report and 70\% Final Exam.
		
	\chapter{Lecture 1: Asset Pricing and Debt Limits under Complete Markets}
	Basic Reference for this part: \texttt{Sargent \& Ljungqvist (2018, Chap. 8)}
	\bb
	Complete markets are an ideal, we study them as a benchmark, from which future frictions cause deviation. 
		
		\section{Section 1:}
		We can think of assets as insurance against consumption shocks. We are interested in the equilibrium prices that arise. 
		
		
		
		\section{Perfect Risk Sharing}
		
		\section{Asset Pricing}
		Note:\\
		\[ q^0_t(s^t) = q\:^{\text{Price at time 0}}_{\text{Which pays me at time t}} \:(s^t)\]
		Remember that:
		\[ \pi(s^0) = q^0_0(s^0) = 1 \]
		and that;
		\[ \beta^t \pi(s^t) u'(c^i_t(s^t)) = \mu^i =q^0_t(s^0) (s^t) \]
		Hence: 
		\[ u'(c^i_t(s^t)) = \mu^i \]
		This means the the price of an AD asset at time 0 is therefore: 
		\[ q^0_t(s^0) = \beta^t \pi(s^t) \dfrac{u'(c^i_t(s^t))}{u'(c^i_0(s^0))} \]
		This ratio is the same for every $i$. 
		\bb
		AD securities are like fundamental particles; all other assets are made of them
		\bb
		We can also imagine therefore an asset which pays a dividend in each time period.  We can still think of them as an AD security, by just adding up the life-time value of the dividend.\\
		It's price in equilibrium must be;
		\[ p^0_0 = \sum_{t=0}^{\infty} \sum_{s^t} q^0_t (s^t)\]
		
		\bb
		\textbf{No Arbitrage Condition:}\\
		According to the no arbitrage condition: Purchasing asset which delivers 1 now costs the same as buying it later.
		\[ q^0_t(s^t) = q^0_\tau(s^\tau)q^\tau_t(s^t) \]
		We call this the term structure of AD securities.\\
		We can use the term structure of AD securities to price any asset at any time $\tau$
		
		
	\end{spacing}
\end{document}
