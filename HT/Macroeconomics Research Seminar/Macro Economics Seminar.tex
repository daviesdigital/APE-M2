\documentclass[11pt]{report}

%use European style
\usepackage[a4paper,left=2cm,right=2cm,top=2cm,bottom=2cm]{geometry}

%few useful packages ------------------------------------------------------------------
\usepackage{setspace}
\let\Tiny=\tiny %remove annoying warnings
\usepackage[english]{babel}
\usepackage[latin1]{inputenc}
\usepackage{amsmath}
\usepackage{amssymb}
\usepackage{amsthm}
\usepackage{amsfonts}
\usepackage{colortbl}
\usepackage{xcolor}
\usepackage{eurosym}
\usepackage{enumitem}
\usepackage{chngpage}
\usepackage{fancyhdr}
\usepackage{fancyvrb}
\usepackage{float}
\usepackage{framed}
\usepackage{multirow}
\usepackage{graphicx}
\graphicspath{ {./images/} }
\usepackage{geometry}
\usepackage{lipsum}
\usepackage{tabularx}
\usepackage[linktocpage]{hyperref}

%define environment for code
\definecolor{orangepse}{RGB}{240,139,39}
\definecolor{redpse}{RGB}{222,6,61}
\newcommand{\rpse}[1]{\textcolor{redpse}{#1}}
\definecolor{dkgreen}{rgb}{0,0.6,0}
\definecolor{gray}{rgb}{0.5,0.5,0.5}
\definecolor{mauve}{rgb}{0.58,0,0.82}

\usepackage{listings}
\lstset{frame=tblr,
	language=R,
	aboveskip=5mm,
	belowskip=5mm,
	showstringspaces=false,
	columns=flexible,
	basicstyle={\small\ttfamily},
	numbers=none,
	numberstyle=\tiny\color{gray},
	keywordstyle=\color{blue},
	commentstyle=\color{dkgreen},
	stringstyle=\color{mauve},
	breaklines=true,
	breakatwhitespace=true,
	tabsize=3
}
%---------------------------------------------------------------------------------------


% New Commands ----------------------------
\newcommand{\bb}{\bigbreak\noindent}

\makeatletter
\renewcommand\section{\leftskip 0pt\@startsection {section}{1}{\z@}%
	{-3.5ex \@plus -1ex \@minus -.2ex}%
	{2.3ex \@plus.2ex}%
	{\normalfont\Large\bfseries}}

\renewcommand\subsection{\leftskip 4ex\@startsection{subsection}{2}{\z@}%
	{-3.25ex\@plus -1ex \@minus -.2ex}%
	{1.5ex \@plus .2ex}%
	{\normalfont\large\bfseries}}

\renewcommand\subsubsection{\leftskip 14ex\@startsection{subsubsection}{3}{\z@}%
	{-3.25ex\@plus -1ex \@minus -.2ex}%
	{1.5ex \@plus .2ex}%
	{\normalfont\large\bfseries}}
\makeatother

%----------------------------------------------------------

\definecolor{titlepagecolor}{cmyk}{1,.60,0,.40}
\definecolor{namecolor}{cmyk}{1,.50,0,.10} 


\begin{document}
	\setcounter{page}{1}
	\begin{spacing}{1.5}
		
		% Table of Contents ----------------------------------
		\tableofcontents
		\setcounter{secnumdepth}{-2}
		\newpage
		
		% Start of Content ---------------------------------------------
		\chapter{General Notes:} 
		
		
		\chapter{Brief Introduction}
		Talk to Giles St-Paul regarding bounded rationality in Macroeconomics.
		\bb
		- Optimum Monetary Policy
		\bb
		- Forward Guidance Problems.
		\begin{itemize}
			\item Inflationary Effects?
			\item Identify the flaw in this model (as low inflation exists)
		\end{itemize}
		\bb
		Heterogeneous Agent Models
		\bb
		Tobias Bohr - Talk to him about him
		\bb
		Bertrand Vignole - Behavioural Macro from a theoretical stand point.
		\bb
		Agnes Benacy - Policy Economist at Bank de France
		\bb
		Antoine Camous - Central Banking Policy.
		
		\section{Section 1:}
		
		\subsection{Section 1:}
		
\chapter{Week 1 Presentations}

	\section{Long-run Labour Supply in Brazil}
	
	\section{Rational bubbles and misallocation}
	
	
\chapter{Macro-days Seminar}
	 \section{Wealth Centric Model of Monetary Policy}
	 Tries to show that wealth effects explain the issues in the theory between interest rates and savings.
	 Reductions in the interest rate seems to be associated with an increase in savings. This does not follow the theory.
	 \bb
	 There seems to exist a non-monotonastic relationship between interest rates and savings.
	 \bb
	 Instead of a positive relationship between $I$ and $S$, they try to model a situation where there is a negative relationship
	 
	 \section{Labour Fluidity and the flattening of the Philips curve}
	 Philips curve slop flattened over time. We can also observe a structural change in the labour market. Can one explain the other? This structural change is down to the increase in non-routine jobs, which are more fluid (Larger Hiring and Separation rate).
	 \bb
	 Find that the occupational shifts of the last two recessions explain more than 25\% of the variation in the Phillips curve.
	 \bb
	 \textbf{Economic Intuition/Motivation: }\\
	 This fluidity, favours the Employer, increasing their bargaining power relative to workers. This means that wages are stickier.
	 \bb
	 
	 \section{Panel Discussion: Structural Changes and its Effect on Monetary Policy}
		We often think of monetary policy as stabilizing demand around some path - let's say growth. 
		\bb
		Don't just look at growth in terms of GDP. There is room to discuss welfare, and observing leisure decisions as a good thing.
		
		\subsection{Three headwinds to growth:}
		\begin{itemize}[leftmargin=10ex]
			\item Are ideas getting harder to find? 
			\begin{itemize}
				\item Ideas come from people, and productivity has tracked population growth.
			\end{itemize}
				
			\item 
		\end{itemize}
		
		\subsection{Three tailwinds to growth}
		
		
		
		
		\section{Thesis topics}
		Heterogeneous Agent Models
		\\
		Productivity of workers - firm and worker specific effects
		
		
\chapter{Monetary Policy Response to Recent Inflation}
\begin{center}
	\large{\textit{The art and science of patience: Relative prices and inflation.}} 
\end{center}
A very doveish report.\\
Closely related to policy discussions.

\bb
Report pushes a third alternative view: 
\begin{itemize}
	\item Aggressive monetary policy may be damaging. 
	\item Analyses whether being an importer of gas and oil is an important factor.
\end{itemize}		

	\section{Inflation}
	4 key facts.
	 
	 \bb
	 Need to explain inflation going down, while core inflation remains persistent...
	 
	 
	 \section{Some Takeaways}
	 uneven shock, big difference in relative prices.
	 \bb
	 Monetary Policy is a blunt instrument. 
	 \begin{itemize}
	 	\item Even more flexible monetary policy  such as "credit support" are not suitable.
	 \end{itemize}
	 \bb
	 Fiscal Policy is much more flexible at resolving these type of issues and filling in the gaps.
	 
	 
	 \section{Thesis Questions}
	 Described as a puzzle: the anchoring of inflation expectations.  
	 
	 \bb
	 Are expectation data even useful? Can this actually guide monetary policy? 
	
	
	
\chapter{Week 3 Presentations:}
	\section{Henri - Corridor vs Floor Operating  System in New Keynesian Model}
		
	\end{spacing}
\end{document}
