
\documentclass[11pt]{article}

%use European style
\usepackage[a4paper,left=2cm,right=2cm,top=2cm,bottom=2cm]{geometry}

%few useful packages ------------------------------------------------------------------
\usepackage{setspace}
\let\Tiny=\tiny %remove annoying warnings
\usepackage[english]{babel}
\usepackage[latin1]{inputenc}
\usepackage{amsmath}
\usepackage{amssymb}
\usepackage{amsthm}
\usepackage{amsfonts}
\usepackage{colortbl}
\usepackage{xcolor}
\usepackage{eurosym}
\usepackage{enumitem}
\usepackage{chngpage}
\usepackage{fancyhdr}
\usepackage{fancyvrb}
\usepackage{float}
\usepackage{framed}
\usepackage{multirow}
\usepackage{graphicx}
\graphicspath{ {./images/} }
\usepackage{geometry}
\usepackage{lipsum}
\usepackage{tabularx}
\usepackage[linktocpage]{hyperref}
\usepackage{graphicx}





\usepackage{listings}
\lstset{frame=tblr,
	language=R,
	aboveskip=5mm,
	belowskip=5mm,
	showstringspaces=false,
	columns=flexible,
	basicstyle={\small\ttfamily},
	numbers=none,
	numberstyle=\tiny\color{gray},
	keywordstyle=\color{blue},
	commentstyle=\color{dkgreen},
	stringstyle=\color{mauve},
	breaklines=true,
	breakatwhitespace=true,
	tabsize=3
}
%---------------------------------------------------------------------------------------


% New Commands ----------------------------
\newcommand{\bb}{\bigbreak\noindent}

\makeatletter
\renewcommand\section{\leftskip 0pt\@startsection {section}{1}{\z@}%
	{-3.5ex \@plus -1ex \@minus -.2ex}%
	{2.3ex \@plus.2ex}%
	{\normalfont\Large\bfseries}}

\renewcommand\subsection{\leftskip 4ex\@startsection{subsection}{2}{\z@}%
	{-3.25ex\@plus -1ex \@minus -.2ex}%
	{1.5ex \@plus .2ex}%
	{\normalfont\large\bfseries}}

\renewcommand\subsubsection{\leftskip 14ex\@startsection{subsubsection}{3}{\z@}%
	{-3.25ex\@plus -1ex \@minus -.2ex}%
	{1.5ex \@plus .2ex}%
	{\normalfont\large\bfseries}}
\makeatother

%----------------------------------------------------------

\definecolor{titlepagecolor}{cmyk}{1,.60,0,.40}
\definecolor{namecolor}{cmyk}{1,.50,0,.10} 

\title{Quantitative macroeconomics - Problem Set 3}
\author{Jorge Aparicio Lopez, Davide Davies-Biletta, Barbara Saget }
\date{October 2023}

\begin{document}
	\setcounter{page}{1}
	\begin{spacing}{1.5}
		
		\maketitle
		
		\section{Problem 1 The Bellman equation and its properties.}
		We simplify the problem as a choice of $\{k_{t+1}\}$: 
		
		$$i_t = k_{t+1}-(1-\delta)k_t$$
		And then:
		$$c_t \leq (1-\delta)k_t + z_t k_{t}^\theta-k_{t+1}$$
		Now the problem simplifies to solving: 
		\[ \max_{\{c_t, k_{t+1}\}} \sum^T_{t=0} \beta ^t \dfrac{c_t^{1-\sigma} - 1 }{1 - \sigma}\]
		\[s.t. \:\:\: c_t \leq (1-\delta)k_t + z_t k_{t}^\theta-k_{t+1} \]
		
		At the optimum the resource constraints will be binding so we can rewrite the program as : 
		\begin{align*}
			&\max_{\{k_{t+1}\}} \sum^T_{t=0} \beta ^t \dfrac{\big((1-\delta)k_t + z_t k_{t}^\theta-k_{t+1}\big)^{1-\sigma} - 1 }{1 - \sigma}  \\
			\iff& \max_{\{k_{t+1}\}} \sum^T_{t=0} \beta ^t F(k_{t+1}, k_t)-1 \quad \text{(SP)} 
		\end{align*}
		with $F(k_{t+1}, k_t)= \dfrac{\big((1-\delta)k_t + z_t k_{t}^\theta-k_{t+1}\big)^{1-\sigma} - 1 }{1 - \sigma} $
		\\
		\\
		Let's define $\Gamma$, X, $\beta$ and F as following:
		\\
		$X=\big\{k_t \in X, k_t \in[0, max(k_0,k*)]\big\}$ \quad set of possible state variable $k_t$
		\\
		$\Gamma : X -> X$ \quad feasible values of the next period capital for a given level of capital stock k.
		\\ 
		The possible values of $k_{t+1}$ have to be such that $k_t\ge 0$ and $c_t\ge 0$. 
		\begin{align*}
			&c_t\ge 0\\
			\iff &(1-\delta)k_t + z_t k_{t}^\theta-k_{t+1} \ge 0 \\
			\iff & k_{t+1}\le (1-\delta)k_t + z_t k_{t}^\theta  
		\end{align*}
		\\
		So $\Gamma=\big\{k_{t+1} \in \Gamma,  k_{t+1} \in[0, (1-\delta)k_t + z_t k_{t}^\theta ]\big\}$ 
		\\
		\\
		$\beta>0$ the constant discount factor
		\\
		\\
		$F(k_{t+1}, k_t)= \dfrac{\big((1-\delta)k_t + z_t k_{t}^\theta-k_{t+1}\big)^{1-\sigma} - 1 }{1 - \sigma} $ with $F:X\times X -> \mathbb{R}$ the one-period return function which maps all feasible combination of today's and tomorrow's capital stock. \color{red}Is F bounded ? \color{black}
		
		Than we can determine the functional equation: 
		
		$v(k_{t+1})= sup_{\{k_{t+1} \in \Gamma(k_t)\}}[F(k_t,k_{t+1}) + \beta k_{t+1}], \quad\text{all}\quad k_t \in X$
		
		\subsection{The return function F is not bounded, but (FE) nevertheless maps C(X), the space of continuous bounded functions on R, into itself. (Make sure you understand why). (FE) defines a contraction mapping using Black-well's sufficient conditions. (Again, make sure you understand why). Can you further characterise v given the functional forms of $f$ and $\Gamma$?}
		
	\end{spacing}
\end{document}