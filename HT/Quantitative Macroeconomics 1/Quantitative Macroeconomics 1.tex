\documentclass[11pt]{report}

%use European style
\usepackage[a4paper,left=2cm,right=2cm,top=2cm,bottom=2cm]{geometry}

%few useful packages ------------------------------------------------------------------
\usepackage{setspace}
\let\Tiny=\tiny %remove annoying warnings
\usepackage[english]{babel}
\usepackage[latin1]{inputenc}
\usepackage{amsmath}
\usepackage{amssymb}
\usepackage{amsthm}
\usepackage{amsfonts}
\usepackage{colortbl}
\usepackage{xcolor}
\usepackage{eurosym}
\usepackage{enumitem}
\usepackage{chngpage}
\usepackage{fancyhdr}
\usepackage{fancyvrb}
\usepackage{float}
\usepackage{framed}
\usepackage{multirow}
\usepackage{graphicx}
\graphicspath{ {./images/} }
\usepackage{geometry}
\usepackage{lipsum}
\usepackage{tabularx}
\usepackage[linktocpage]{hyperref}

%define environment for code
\definecolor{orangepse}{RGB}{240,139,39}
\definecolor{redpse}{RGB}{222,6,61}
\newcommand{\rpse}[1]{\textcolor{redpse}{#1}}
\definecolor{dkgreen}{rgb}{0,0.6,0}
\definecolor{gray}{rgb}{0.5,0.5,0.5}
\definecolor{mauve}{rgb}{0.58,0,0.82}

\usepackage{listings}
\lstset{frame=tblr,
	language=R,
	aboveskip=5mm,
	belowskip=5mm,
	showstringspaces=false,
	columns=flexible,
	basicstyle={\small\ttfamily},
	numbers=none,
	numberstyle=\tiny\color{gray},
	keywordstyle=\color{blue},
	commentstyle=\color{dkgreen},
	stringstyle=\color{mauve},
	breaklines=true,
	breakatwhitespace=true,
	tabsize=3
}
%---------------------------------------------------------------------------------------


% New Commands ----------------------------
\newcommand{\bb}{\bigbreak\noindent}

\makeatletter
\renewcommand\section{\leftskip 0pt\@startsection {section}{1}{\z@}%
	{-3.5ex \@plus -1ex \@minus -.2ex}%
	{2.3ex \@plus.2ex}%
	{\normalfont\Large\bfseries}}

\renewcommand\subsection{\leftskip 4ex\@startsection{subsection}{2}{\z@}%
	{-3.25ex\@plus -1ex \@minus -.2ex}%
	{1.5ex \@plus .2ex}%
	{\normalfont\large\bfseries}}

\renewcommand\subsubsection{\leftskip 14ex\@startsection{subsubsection}{3}{\z@}%
	{-3.25ex\@plus -1ex \@minus -.2ex}%
	{1.5ex \@plus .2ex}%
	{\normalfont\large\bfseries}}
\makeatother

%----------------------------------------------------------

\definecolor{titlepagecolor}{cmyk}{1,.60,0,.40}
\definecolor{namecolor}{cmyk}{1,.50,0,.10} 


\begin{document}
	\setcounter{page}{1}
	\begin{spacing}{1.5}
		
		% Table of Contents ----------------------------------
		\tableofcontents
		\setcounter{secnumdepth}{-2}
		\newpage
		
		% Start of Content ---------------------------------------------
		\chapter{General Notes:} 
		70\% Problem Sets - 30\% Small Exam 
		
		\section{Reference Book}
		Heer \& Maussner - Dynamic General Equilibrium Modelling 
		
		\section{Main elements of a model economy}
		\begin{itemize}
			\item Agents
			\begin{itemize}
				\item Households and Firms
			\end{itemize}
			\item Goods
			\begin{itemize}
				\item Consumption Good
				\item Investment Good
			\end{itemize}
			\item Time
			\begin{itemize}
				\item Discreet vs continuous
			\end{itemize}
			\item Structure of Uncertainty
			\begin{itemize}
				\item Shocks to the economy 
			\end{itemize}
			
			\item Structure of Market
			\begin{itemize}
				\item Competition vs.  oligopoly vs. monopoly
			\end{itemize}
		\end{itemize}
		
		
		
		
		\chapter{Lecture 1: Basic concepts and methods in numerical analysis}
		Solivng problems Numerically:
		\begin{itemize}
			\item  Solve the \textit{Policy} function
			\item Aggregate
			\begin{itemize}
				\item Easy for Representative Agent models 
				\item Harder for Heterogeneous Agent models
			\end{itemize}
			\item Solve for prices such that markets clear
		\end{itemize}
		
		\section{Types of Error}
		\begin{itemize}
			\item Truncation
			\item Round Off
		\end{itemize}
		
		\section{Solving Linear Equations}
		
		\section{Solving Non-Linear Equations}
		
			\subsection{Bisection}
			
			\subsection{Newton Methods}
			
				\subsubsection{Secant Method}
				
			
			
			\subsection{Golden Ratio}
		
		
		
		\chapter{Lecture 2: Solving deterministic dynamic models}
		
		\section{Ramsey ``Neoclassical" Growth Model}
		
			\subsection{Environment}
			\begin{itemize}[leftmargin=10ex]
				\item Agents: A representative firm, and a representative HH.
				\item Time: t = 1, 2, ..., no uncertainty.
				\item Goods: Labour service lt , a single numeraire output good (price normalised
				to 1) used for consumption $c_t$ and investment it , and a capital good Kt .
				\item Endowments: HH is endowed with one unit of labour each period, and
				initial capital $k_0$.
				\item Technology:
				\begin{itemize}
					\item The Firm has access to a production function $F (k_t , l_t , z_t )$, including a
					productivity parameter zt .
					\item Households have access to a capital accumulation technology:
					$k_{t+1} = (1 ? \delta)k_t + i_t$ (where it may be negative).
				\end{itemize} 
				\item Preferences: Household preferences are defined over consumption and
				hours U = ??
				t=0 ?t u(ct , lt )
				\item Ownership: The HH owns the firm, receives its profits from production.
				\item Market structure:
				\begin{itemize}
					\item ``Sequential" trade: Every period, agents trade the numeraire good, the
					labor service lt (at price wt ) and capital services (whose ?rental price" is rt ).
					\item Firm and HH behave competitively, i.e. maximise their objectives taking
					prices wt and $r_t$ as given. 
				\end{itemize}
			\end{itemize}
			
			\subsection{Planner's Problem}
			Simplify using assumptions.
			
			
			
			\subsubsection{Does a solution exist?}
			Well K lies in a domain.
			\begin{itemize}[leftmargin=15ex]
				\item MPK rises at a depreciating rate
				\item Cost of Capital rises linearly
				\item the points at which they meet (0 - $\bar{F}(K^{max})$ outlines max and min)
			\end{itemize}
			
			\subsubsection{Uniqueness}
			
			
			\subsection{Phase diagram}
			Trivially, we can rule out solutions other than the steady state due to the \textit{Transversality Condition}.
			
			\section{How to find a numerical solution for more general $u, F , \theta$?}
			
			\begin{itemize}
				\item Linearisation
				\item Break up the problem in separate sub-problems: \textit{Dynamic programming}
					\begin{itemize}
						\item Solving directly the sequences
					\end{itemize}
			\end{itemize}
		
		
\part{Tutorials}

\chapter{Tutorial 1: }


\chapter{Tutorial 2}
Interpolation is 


\[ interp1( \text{\textit{x vector of points we know} },\\
 \text{\textit{Value of the function at known points}} , \\ 
 \textit{ Value of function at points we dont know}) \]
		
		
		
		

		
		
	\end{spacing}
\end{document}
