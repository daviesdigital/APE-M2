\documentclass[11pt]{report}

%use European style
\usepackage[a4paper,left=2cm,right=2cm,top=2cm,bottom=2cm]{geometry}

%few useful packages ------------------------------------------------------------------
\usepackage{setspace}
\let\Tiny=\tiny %remove annoying warnings
\usepackage[english]{babel}
\usepackage[latin1]{inputenc}
\usepackage{amsmath}
\usepackage{amssymb}
\usepackage{amsthm}
\usepackage{amsfonts}
\usepackage{colortbl}
\usepackage{xcolor}
\usepackage{eurosym}
\usepackage{enumitem}
\usepackage{chngpage}
\usepackage{fancyhdr}
\usepackage{fancyvrb}
\usepackage{float}
\usepackage{framed}
\usepackage{multirow}
\usepackage{graphicx}
\graphicspath{ {./images/} }
\usepackage{geometry}
\usepackage{lipsum}
\usepackage{tabularx}
\usepackage[linktocpage]{hyperref}

%define environment for code
\definecolor{orangepse}{RGB}{240,139,39}
\definecolor{redpse}{RGB}{222,6,61}
\newcommand{\rpse}[1]{\textcolor{redpse}{#1}}
\definecolor{dkgreen}{rgb}{0,0.6,0}
\definecolor{gray}{rgb}{0.5,0.5,0.5}
\definecolor{mauve}{rgb}{0.58,0,0.82}

\usepackage{listings}
\lstset{frame=tblr,
	language=R,
	aboveskip=5mm,
	belowskip=5mm,
	showstringspaces=false,
	columns=flexible,
	basicstyle={\small\ttfamily},
	numbers=none,
	numberstyle=\tiny\color{gray},
	keywordstyle=\color{blue},
	commentstyle=\color{dkgreen},
	stringstyle=\color{mauve},
	breaklines=true,
	breakatwhitespace=true,
	tabsize=3
}
%---------------------------------------------------------------------------------------


% New Commands ----------------------------
\newcommand{\bb}{\bigbreak\noindent}

\makeatletter
\renewcommand\section{\leftskip 0pt\@startsection {section}{1}{\z@}%
	{-3.5ex \@plus -1ex \@minus -.2ex}%
	{2.3ex \@plus.2ex}%
	{\normalfont\Large\bfseries}}

\renewcommand\subsection{\leftskip 4ex\@startsection{subsection}{2}{\z@}%
	{-3.25ex\@plus -1ex \@minus -.2ex}%
	{1.5ex \@plus .2ex}%
	{\normalfont\large\bfseries}}

\renewcommand\subsubsection{\leftskip 14ex\@startsection{subsubsection}{3}{\z@}%
	{-3.25ex\@plus -1ex \@minus -.2ex}%
	{1.5ex \@plus .2ex}%
	{\normalfont\large\bfseries}}
\makeatother

%----------------------------------------------------------

\definecolor{titlepagecolor}{cmyk}{1,.60,0,.40}
\definecolor{namecolor}{cmyk}{1,.50,0,.10} 


\begin{document}
	\setcounter{page}{1}
	\begin{spacing}{1.5}
		
		% Table of Contents ----------------------------------
		\tableofcontents
		\setcounter{secnumdepth}{-2}
		\newpage
		
		% Start of Content ---------------------------------------------
		\chapter{General Notes:} 
		Problem Set - 33\%\\
		Final Exam - 66\%
		\begin{itemize}
			\item Part problem set
			\item Part on papers/slides
		\end{itemize}
		
	
		\chapter{Lecture 1: Introduction}
		History of the subject:\\
		Political Economy (18th Century) $\rightarrow$ Public Finance (19th Century) $\rightarrow$ Public Economics (1960s)
		
		\section{History}
		\subsection{History - 18th Century}
		Adam Smith
		
		\subsection{History - 19th Century}
		\begin{itemize}[leftmargin=10ex]
			\item German School:
			\begin{itemize}
				\item Economics more favourable to the public sector
			\end{itemize}
			\item French School
			\begin{itemize}
				\item Focus on infrastructural spending (Bridges, railroads)
			\end{itemize}
			\item Italian School
			\item Stockholm School
		\end{itemize}
		
		
		\subsection{Two types of questions}
			\begin{itemize}[leftmargin=10ex]
				\item Positive Approach
				\begin{itemize}
					\item How can economic policy help?
				\end{itemize}
				\item Normative Approach
				\begin{itemize}
					\item How should economic policy be enacted to attain a certain desired social goal?
				\end{itemize}
			\end{itemize}
			
			\section{Spending Money}
			\subsection{Growth of the State}
			In the 19th Century governments were minimal
			\begin{itemize}[leftmargin=10ex]
				\item Spending made up approx 10\% of GDP
				\item Mostly on Police and Military
				\item Almost no Social Spending
			\end{itemize}
			\bb
			In the 20th Century the spending grew
			\begin{itemize}[leftmargin=10ex]
				\item Spending now makes up on average approx. 45\% of GDP
				\item Huge increase in social spending
				\item This is true for developed countries but not so in less developed 
			\end{itemize}
			
			\subsection{The Rise of Social State}
			\begin{itemize}[leftmargin=10ex]
				\item Bismarck Social Insurance (1883)
				\item Beveridge National Insurance (1941)
				\item S�curit� Sociale (1945)
			\end{itemize}
			Public social insurance spending explains a large part of overall differences in public spending (notably pension spending)
			
			\section{Raising Money}
			\subsection{Taxation}
			Logically, as state spending grows (as a share of GDP), so does Taxation.
			
			\bb
			Headline tax burden statistics can hide nuanced differences similarities between countries.
			\begin{itemize}[leftmargin=10ex]
				\item France has a lower tax burden that spending per GDP
				\begin{itemize}
					\item They make money from state ownership of companies which adds to their revenue
				\end{itemize}
			\end{itemize}
			
			\subsection{Structure of Revenue}
			5 main components: 
			\begin{itemize}[leftmargin=10ex]
				\item Personal Income Tax
				\item Corporate Income Tax
				\item Social Security Contributions
				\item Consumption Tax
				\item Property Tax
			\end{itemize}
			
			\section{Growth of the State Literature}
			\textbf{Wagner:}\\
			Demand for Public Goods grows with income. (Elasticity $>$ 1)
			
			\bb
			\textbf{Baumol's cost disease: }\\
			Economics of performing arts $\rightarrow$ Productivity increases in the private sector mean that the cost to provide labour intensive public services increases (leading to higher spending)
			\begin{itemize}[leftmargin=10ex]
				\item Not that convincing to explain the large increase
			\end{itemize}
			
			\bb
			\textbf{Ratchet Effect Theory:}\\
			Wars increase government spending and taxation, which was not then reversed after these events.
			
			\bb
			\textbf{Leviathan Theory}\\
			Governments are controlled by self-interested in politician-bureaucrats
			\begin{itemize}[leftmargin=10ex]
				\item Also shit
			\end{itemize}
			
			\bb
			\textbf{Political Economy}\\
			As states become more democratic, the poor are better represented and vote for greater public spending.
			\begin{itemize}[leftmargin=10ex]
				\item Chicken or egg theory?
			\end{itemize}
			
			
				
			\subsection{Fundamental Theorems of Welfare Economics}
	
	\chapter{Lecture 2: Tools of Welfare Analysis}
	
		\section{The Concept of Economic Surplus}
		The amount by which buyers and sellers benefit from participating in the market.
		
			\subsection{Consumer Surplus}
			Harder to calculate consumer surplus because we need measure a monetary equivalent of utility. To do this we find the difference between:
			\begin{center}
				What someone \textit{would} pay for a good and what someone \textit{did} pay for a good
			\end{center}
			\bb
			Two exact measures of changes in consumer welfare: 
			\begin{itemize}[leftmargin=10ex]
				\item Compensating variation
				\item Equivalent variation
			\end{itemize}
			\bb
			We use \textbf{Quasi-Linear Utility}
			\begin{itemize}[leftmargin=10ex]
				\item Only appropriate when the good:
				\begin{itemize}
					\item Costs a small percentage of income (Income effects are negligible)
					\item Has no close substitutes
				\end{itemize}
			\end{itemize}
			This choice of model is a significant restriction.
			
				\subsubsection{Model}
				
				\textbf{NB:}
				Marshallian Demand: Demand as a function of price and income\\
				Hisksian Demand: Demand as a function of price and utility.
				
				
				
				\subsubsection{Willingness to Pay (WTP)}
				\[ U(x_0, m - WTP) = U(0,m) \]
				\[ WTP = v(x_0) - v(0) = \int \]
				
				
				\subsubsection{Consumer Surplus Equation}
				
				
			\subsection{Equivalent and Compensating Variation}
			\textbf{Problem:} Changes in  CS are and exact measure of consumer's welafare change only if utility is quasilinear (no income effects).
			\bb
			\textbf{Compensating Variation:} \\
			Amount of money that must be taken away from consumer after the change to restore her original utility level. 
			\begin{itemize}[leftmargin=10ex]
				\item Willingness to Pay
			\end{itemize}
			\bb
			\textbf{Eqivalent Variation:}\\
			Amount of money that must be given to the consumer before the change to leave her as well off as the change.
			\begin{itemize}[leftmargin=10ex]
				\item Willingness to Accept
			\end{itemize}
			
			\bb
			The value of income is not the same 
			
			\bb
			\begin{framed}
				For a \textit{\textbf{Normal Good}}:
				\[ CV < \Delta CS < EV \]
			\end{framed}
			
		
			
			
			\subsection{Producer Surplus}
			Naturally in monetary terms $\rightarrow$ easier to calculate
			
			\begin{center}
				Willingness to Sell - How much you receive
			\end{center} 
			\bb
			Closely related to profit:
			\[ Producer \; Surplus = Profit + Fixed \: Costs \]
			Firms produce output only if the producer surplus is positive. If not they shut down.
		
		\section{Competitive Equilibrium and Social Efficiency}
		
		\begin{framed}
			\noindent
			\textbf{The First Theorum of Welfare }\\
			The competitive equilibrium maximises social efficiency.
		\end{framed}
		
		
		
		
		\section{Measuring Inefficiency: Deadweight Loss}
		Cause the market economu to deliver outcomes that do not maximise efficiency 
		\begin{itemize}[leftmargin=10ex]
			\item Imperfect competition
			\item Public Goods
			\item Externalities
			\item Asymetric Info
		\end{itemize}
		Can also result from Government intervention
		
			\subsubsection{Loss of Monopoly Pricing}
			
			\subsubsection{Rent Control}
			Leads to shortages.
			\bb
			\textbf{N.B.} This graph assumes that people are getting the house recieve the in order of their marginal valuation.
			\bb
			However, if you assume housing is randomly allocated:\\
			There is a potentially an extra cost to consumers in the form of lost consumer surplus due to misallocation.
		
		
		\section{The Efficiency Cost of Taxation}
		The government raises taxes for two main reasons
		\begin{itemize}
			\item Raise revenue to finance government expenditure 
			\item redistribute income
		\end{itemize}
		\bb
		\textbf{Ideally:} Use lump-sum taxes that don't lead to changes in peoples behaviour (ie. a tax on height). \\
		These are not feasible and so Govs must rely on \textbf{distortionary} taxes. 
		\begin{center}
			Goal: How do we minimise these distortionary effects?
		\end{center}
		

		\section{LOOK AT PAPERS EXAMPLE IN SLIDES}
		
	\chapter{Externalities}
		
		An externality is a situation where the action of one party directly makes another part worse off/better off, yet the first party does not bare the cost nor benefits of doing so.
		\begin{itemize}
			\item Important distinction between pecuniary and non-pecuniary
		\end{itemize}
		\bb
		\begin{tabularx}{\textwidth}{|X|X|}
			\hline
		     &  \\
		     \hline
		     
		\end{tabularx}
		
		\section{Social Optimum}
		
		\begin{center}
			Private Marginal Benefit = Private Marginal Cost + Marginal Cost
		\end{center}
		
		\bb Take-away lessons: 
		\begin{itemize}
			\item Private markets do not produce Pareto efficient outcomes because firms/consumers do not take into account marginal damage (MD) or external marginal benefit (EMB) of production/consumption
			
			\item Completely eliminating the externality is not necessarily desirable
			
			\item Need to know the shapes of PMB, PMC and MD/EMB to implement the socially optimal level
		\end{itemize}

		\section{Measuring Externalities}
		By definition, no direct market can be used to recover WTP to reduce negative externalities / increase positive externalities (if there were a market, there would be no externality!)
		
			\subsection{Direct Valuation}
			Directly measure the physical effects of externalities (e.g., material deteriorations, damages to health, etc.) and use existing market prices (e.g., medical expenses, wages) to assign a monetary value to these physical effects. 
			
			\bb
			\textbf{Problem:} individuals may compensate for increases in pollution by reducing their exposure, resulting in estimates that understate the full welfare costs of air pollution (avoidance behaviour is costly!)
			
				\subsubsection{Moretti and Neidell (2011)}
				The use daily variations in ozone levels due to boat arrivals in two major LA ports as instrument for air pollution; because boat traffic is generally unobserved by local residents, it is assumed to be uncorrelated with pollution avoidance behaviour.
				
				\subsubsection{Limitations}
				\begin{itemize}[leftmargin=20ex]
					\item Direct-cost valuation of externalities is difficult to implement in
					practice:
					\begin{itemize}
						\item requires identifying all the channels through which an
						externality may affect welfare (e.g., damages caused by
						nuclear leakage?)
					\end{itemize}
					\begin{itemize}
						\item market prices not always available to assign a monetary value
						to external effects (e.g., cost of noise?)
					\end{itemize}
				\end{itemize}
				
				
				
			\subsection{Contingent Valuation}
			Ask people directly about their willingness-to-pay (WTP)
			
				\subsubsection{Limitations}
				\begin{itemize}[leftmargin=20ex]
					\item Cost of designing and conducting survey
					\item Framing Effects (Ex: whales then seals vs. seals then whales)
					\item ``Embedding'' effects (Ex: WTP to clean one lake = WTP to clean 5 lakes)
					\item Strategic responses
				\end{itemize}
				
			\subsection{Hedonic Valuation}
			Method for estimating the value of a traded good: decompose the good (e.g., a house) into a set of
			characteristics (e.g., size, local amenities, pollution) and estimate the specific contribution of each characteristic to the overall value.
			\begin{itemize}
				\item That way you can find out to what extent pollution level contributes to housing prices
				\item Allowing you to reveal their willingness-to-pay for lower pollution.
			\end{itemize}
				
				
				\subsubsection{Chay and Greenstone (2005)}
			

			
		\section{Correcting Externalities}
		
			\subsection{Private Solutions}
				\subsubsection{Coasian Bargaining}
				Externalities emerge because property rights are not well defined. Hence, all we need to do is establish property rights to create markets for pollution.
				\bb
				Two-parts:
				\begin{enumerate}[leftmargin=20ex]
					\item With well-defined property rights and \textbf{\textit{costless bargaining}}, negotiations between the parties creating/affected by the externality can bring about the social optimum.
					\item The efficient solution does not depend on how property rights are assigned.
					\begin{itemize}
						\item This only effects distribution but not efficiency
					\end{itemize}
				\end{enumerate}
				\bb
				However: The assumption of costless bargaining is very strict and rarely, if ever, holds.
				
			\subsection{Public Solutions}
				\subsubsection{Command-and-control}
				\subsubsection{Market-Based}











			
\part{Reading}
	
	\chapter{Tools of Welfare Analysis}
	
		\section{Auerbach, A. (1985). ``The Theory of Excess Burden and Optimal Taxation"}
		
		
		
		\section{Cohen, P., Hahn, R., Hall, J., Levitt, S. and Metcalfe, R. (2016). ``Using Big Data to Estimate Consumer Surplus: The Case of Uber"}
		
		
		\section{Marion, J. and Muehlegger, E. (2008). ``Measuring Illegal Activity and the Effects of Regulatory Innovation: Tax Evasion and the Dyeing of Untaxed Diesel"}
		
			
			
		
		
	\end{spacing}
\end{document}