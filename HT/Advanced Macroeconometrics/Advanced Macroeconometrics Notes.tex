\documentclass[11pt]{report}

%use European style
\usepackage[a4paper,left=2cm,right=2cm,top=2cm,bottom=2cm]{geometry}

%few useful packages ------------------------------------------------------------------
\usepackage{setspace}
\let\Tiny=\tiny %remove annoying warnings
\usepackage[english]{babel}
\usepackage[latin1]{inputenc}
\usepackage{amsmath}
\usepackage{amssymb}
\usepackage{amsthm}
\usepackage{amsfonts}
\usepackage{colortbl}
\usepackage{xcolor}
\usepackage{eurosym}
\usepackage{enumitem}
\usepackage{chngpage}
\usepackage{fancyhdr}
\usepackage{fancyvrb}
\usepackage{float}
\usepackage{framed}
\usepackage{multirow}
\usepackage{graphicx}
\graphicspath{ {./images/} }
\usepackage{geometry}
\usepackage{lipsum}
\usepackage{tabularx}
\usepackage[linktocpage]{hyperref}

%define environment for code
\definecolor{orangepse}{RGB}{240,139,39}
\definecolor{redpse}{RGB}{222,6,61}
\newcommand{\rpse}[1]{\textcolor{redpse}{#1}}
\definecolor{dkgreen}{rgb}{0,0.6,0}
\definecolor{gray}{rgb}{0.5,0.5,0.5}
\definecolor{mauve}{rgb}{0.58,0,0.82}

\usepackage{listings}
\lstset{frame=tblr,
	language=R,
	aboveskip=5mm,
	belowskip=5mm,
	showstringspaces=false,
	columns=flexible,
	basicstyle={\small\ttfamily},
	numbers=none,
	numberstyle=\tiny\color{gray},
	keywordstyle=\color{blue},
	commentstyle=\color{dkgreen},
	stringstyle=\color{mauve},
	breaklines=true,
	breakatwhitespace=true,
	tabsize=3
}
%---------------------------------------------------------------------------------------


% New Commands ----------------------------
\newcommand{\bb}{\bigbreak\noindent}

\makeatletter
\renewcommand\section{\leftskip 0pt\@startsection {section}{1}{\z@}%
	{-3.5ex \@plus -1ex \@minus -.2ex}%
	{2.3ex \@plus.2ex}%
	{\normalfont\Large\bfseries}}

\renewcommand\subsection{\leftskip 4ex\@startsection{subsection}{2}{\z@}%
	{-3.25ex\@plus -1ex \@minus -.2ex}%
	{1.5ex \@plus .2ex}%
	{\normalfont\large\bfseries}}

\renewcommand\subsubsection{\leftskip 14ex\@startsection{subsubsection}{3}{\z@}%
	{-3.25ex\@plus -1ex \@minus -.2ex}%
	{1.5ex \@plus .2ex}%
	{\normalfont\large\bfseries}}
\makeatother

%----------------------------------------------------------

\definecolor{titlepagecolor}{cmyk}{1,.60,0,.40}
\definecolor{namecolor}{cmyk}{1,.50,0,.10} 


\begin{document}
	\setcounter{page}{1}
	\begin{spacing}{1.5}
		
		% Table of Contents ----------------------------------
		\tableofcontents
		\setcounter{secnumdepth}{-2}
		\newpage
		
		% Start of Content ---------------------------------------------
		\chapter{General Notes:} 
		
		\subsection{Terminology:}
		\begin{itemize}
			\item Stochastic Process = Stochastic Model 
			
		\end{itemize}
		
		\chapter{Basics:}
		
			\subsection{GDP}
			Real GDP increasing since WWII due to 
			\begin{itemize}[leftmargin=10ex]
				\item Capital
				\item Productivity 
				\item Population
				\begin{itemize}
					\item Both more labour and more consumers
				\end{itemize}
			\end{itemize}
			Recession = Drop in the level of production (Real GDP)\\
			The process of Recessions and expansions after recessions is known as the Business Cycle
			\bb
			We're going to focus on GDP Growth Rate.
			
			
			\subsection{Unemployment Rate}
			Unlike Real GDP, there is no trend.
			\begin{itemize}[leftmargin=10ex]
				\item This is not a production function but a ratio
			\end{itemize}
			\bb
			\textbf{Great proxy for the US business cycle}
			\begin{itemize}
				\item Very flexible labour market in the US 
			\end{itemize}
			
			\subsection{Interest Rate}
			Short term vs Long Term. 
			\begin{itemize}[leftmargin=10ex]
				\item The central bank decides the short term rate.
				\item The market decides the long term rate.
			\end{itemize} 

				\subsubsection{The Sovereign Rate}
				The rate at which Governments are able to borrow money. 
				\begin{itemize}[leftmargin=20ex]
					\item Usually look at the 10 year rate
					\begin{itemize}
						\item The average period of maturity for government issued bonds
					\end{itemize}
				\end{itemize}
				
			\section{Definitions}
			\begin{framed}
				\noindent \textbf{Stochastic Process:}\\
				$(X_t)_t \in Z$is a sequence of random variables,
				taking real values, indexed by $t \in Z$
				\begin{itemize}
					\item Our (very ambitious) objective: \\
					Find the best process $(X_t)$ that generated $x_t$
					\vspace{-2ex}
					\begin{itemize}
						\item Start with results, end with model
					\end{itemize}
				\end{itemize}
			\end{framed}
			\bb
			\textbf{Simulation:} When you create a model and generate outcomes. (Start with model, end with results)
			
			\subsection{Process}
			The main characteristic of a trajectory $(x_1,...,x_T)$ steming from a stochastic process is the non-independence of the variables.
			\begin{itemize}[leftmargin=10ex]
				\item The first i of the usual i.i.d. hypothesis is no more valid!
				\item Standard tool for measuring dependence: \textbf{linear correlation coefficient}
			\end{itemize}
			
			\subsection{Autocorrelation}
			
			\subsection{Partial Autocorrelation}
			But what if we want to find the autocorrelation between two particular points in the trend and not the trend as a whole? 
			\begin{itemize}[leftmargin=10ex]
				\item Partial Autocorrelation
			\end{itemize}
			
			\section{Identifying a candidate process}
			ACF provides a measure of the persistence of the process or its \textbf{\textit{memory}}
			\bb
			Starting from this information, we will search for a type of process able to fit this persistence.
			\bb
			3 types: 
			\begin{itemize}
				\item No Memory
				\item Short Memory
				\item Long Memory
			\end{itemize}
			
			\subsection{No Memory - White Noise Process}
			No autocorrelation (no dependent). \\
			The value of tomorrow is completely uncorrelated with value of today.
			\begin{itemize}[leftmargin=10ex]
				\item \textbf{Careful:} This does not mean they are independent
			\end{itemize}
			
			\bb
			\begin{center}
			\textbf{EXAMPLE:} Exchange Rates
			\end{center}
			\bb
				\subsubsection{Weak white noise:}
				\[ E(XY) = 0 \text{  given  } \mu(X,Y) = 0 \]
		
				 but maybe: There is a correlation between their variances
				 \[ E(X^2Y^2) \neq 0  \]
				
				\subsubsection{Strong White Noise:}
				Not only \textbf{non-correlated} but also \textbf{fully independent} from one another.
				
				
			\subsection{Short Memory}
			A stochastic process is said to be short-memory is its ACF is such that:
			\[ \rho(k) \leq C_{r^k} , k \rightarrow \infty \], 
			where $C > 0, 0 < r < 1 \text{  and  } k = 1, 2, . . .$
			
				\subsubsection{Examples:}
				AR = Autoregressive.
				
			\subsection{Long Memory}
			When the ACF is non null for large $k$, a.k.a strongly persistent
			\begin{center}
				-------------- Not covered in this course --------------
			\end{center}		
			
			
			\section{Stationarity}
			While time series are non-independent, are they identically distributed?
				\subsection{Strong Stationarity}
				Too strict and complicated to work out in real life
				
				\subsection{Weak Stationarity}
				\begin{itemize}[leftmargin=10ex]
					\item Mean is constant over time
					\item The covariance is constant over time
				\end{itemize}
				\bb
				These properties allow us to estimate the mean by using the empirical mean.
				\[ \hat  {\mu} = \bar{X_T} \]
				\bb
				Therefort he natural predictor of a stationary process is the mean.
				
				
			\section{Wold Theorem}
			
				
			
			\section{Defining a \textit{good} predictor}
			\begin{itemize}
				\item Unbias
				\item Minimum variance
				\begin{itemize}
					\item Otherwise you get large standard errors and uncertainty
				\end{itemize}
			\end{itemize}
			
				\subsection{Forcasting Error Process}
				\[ e_{t+h} = X_{t+h} - \hat{X}_t(h) \]
				
					\subsubsection{Criteria to measure the forcasting accuracy}
					\[ Mean \: Error = E(e_{T+h}) \] 
					\begin{itemize}[leftmargin=15ex]
						\item Might not be bias but can hide inaccuracy
					\end{itemize}
					\bb
					\[ Mean \: Absolute \: Error = E(|e_{T+h}|) \] 
					\begin{itemize}[leftmargin=15ex]
						\item Cannot differentiate at Zero
					\end{itemize}
					\bb
					\begin{framed}
						
					\[ Mean \: Squared \: Error = E(e^2_{T+h}) \]
					\begin{itemize}[leftmargin=10ex]
						\item Allows us to differentiate!
					\end{itemize}
					
					\[ Root \: Mean \: Squared \: Error = \sqrt{E(e^2_{T+h})} \]
					\begin{itemize}[leftmargin=10ex]
						\item Allows us to differentiate!
						\item Values are similar to those observed
					\end{itemize}
					
					\end{framed}
					
					
					
			\section{Linear Process}
			\[ X_t = \sum^{\infty}_{i=0}a_i\varepsilon_i \]
			where: 
			\begin{itemize}
				\item coefficients  $ a_i$ is absolutely summable (finite)
				\item $(\varepsilon_t)_t$ is strict white noise
			\end{itemize}
			
			
			\section{Best Predictor}
			The predictor $\hat{X}_T(h) $ that minimises the MSE is the least squares predictor.
			\[ \hat{X}_T(h) = E(X_{T+h} | I_T) \]
			
			
			\subsection{Distribution of the Predictor}
			 For any linear process, we get $E(e_{T+h}) = 0 $, and;
			\[  E(e^2_{T+h}) = \sigma^2_\varepsilon \sum_{i=0}^{h-1}a^2_i \], where $a_0 = 1$.
			
			\bb
			Normal Distribution:
			
			
			\subsection{Example}
			
			$\phi$ represents the strength of the dependence between $X_t$ and $X_{t-1}$ 
			
			\bb
			\[ X_{t+1} = \phi X_t + \varepsilon{t+1} \]
			\[ E(X_{t+1} | I_t) = E(\phi X_t + \varepsilon{t+1} | I_t)\]
			\[ E(X_{t+1} | I_t) = E(\phi X_t| I_t) + (\varepsilon{t+1} | I_t)\]
			\[ E(X_{t+1} | I_t) = \phi X_t + 0\]
			\[ \hat{X}_{t+1} = \phi X_t  \]
			
			\bb
			Example 2:
			\[ X_{t}(2) = \phi X_{t+1} + \varepsilon{t+2} \]
			\[ E(X_{t+2} | I_t) = E(\phi X_{t+1} + \varepsilon{t+2} | I_t)\]
			\[ E(X_{t+1} | I_t) = E(\phi X_{t+1}| I_t) + (\varepsilon{t+2} | I_t)\]
			\[ E(X_{t+1} | I_t) = \phi X_{t+1} + 0\]
			\[ \hat{X}_{t+1} = \phi (\phi X_t)  \]
			\[ \hat{X}_{t+1} = \phi^2 X_t  \]
			
			
			
			\section{Impulse Response Function (IRF)}
			Objective: Assess the impact of a shock at a given date on the process dynamics.
			
			\bb
			General Definition:
			 \[ GIRF = E(X_{t+h} | I_t, \varepsilon_t = \delta, \varepsilon_s = 0, s > t) - E(X_{t+h} | I_t, \varepsilon_s = 0, s \geq t) \]
			 \begin{center}
			 	GIRF = Whats going on with impulse - What would be going on without impulse
			 \end{center}
			
			
			
			\section{ARMA Models}
			
			
			
			
			
			
			
			
			
			
			
			
			
			
			
			
			
	\end{spacing}
\end{document}
