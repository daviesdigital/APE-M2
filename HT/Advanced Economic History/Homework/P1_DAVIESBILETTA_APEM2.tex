\documentclass[10pt]{article}

%use European style
\usepackage[a4paper,left=2cm,right=2cm,top=2cm,bottom=2cm]{geometry}

%few useful packages ------------------------------------------------------------------
\usepackage{setspace}
\let\Tiny=\tiny %remove annoying warnings
\usepackage[english]{babel}
\usepackage[latin1]{inputenc}
\usepackage{amsmath}
\usepackage{amssymb}
\usepackage{amsthm}
\usepackage{amsfonts}
\usepackage{colortbl}
\usepackage{xcolor}
\usepackage{eurosym}
\usepackage{enumitem}
\usepackage{chngpage}
\usepackage{fancyhdr}
\usepackage{fancyvrb}
\usepackage{float}
\usepackage{framed}
\usepackage{multirow}
\usepackage{graphicx}
\graphicspath{ {./images/} }
\usepackage{geometry}
\usepackage{lipsum}
\usepackage{tabularx}
\usepackage[linktocpage]{hyperref}

\usepackage{tikz}
\usetikzlibrary{positioning}

%define environment for code
\definecolor{orangepse}{RGB}{240,139,39}
\definecolor{redpse}{RGB}{222,6,61}
\newcommand{\rpse}[1]{\textcolor{redpse}{#1}}
\definecolor{dkgreen}{rgb}{0,0.6,0}
\definecolor{gray}{rgb}{0.5,0.5,0.5}
\definecolor{mauve}{rgb}{0.58,0,0.82}

\usepackage{listings}
\lstset{frame=tblr,
	language=R,
	aboveskip=5mm,
	belowskip=5mm,
	showstringspaces=false,
	columns=flexible,
	basicstyle={\small\ttfamily},
	numbers=none,
	numberstyle=\tiny\color{gray},
	keywordstyle=\color{blue},
	commentstyle=\color{dkgreen},
	stringstyle=\color{mauve},
	breaklines=true,
	breakatwhitespace=true,
	tabsize=3
}
%---------------------------------------------------------------------------------------

\usepackage[
backend=biber,
style=apa
]{biblatex}
\addbibresource{bibp1.bib}

% New Commands ----------------------------
\newcommand{\bb}{\bigbreak\noindent}

\makeatletter
\renewcommand\section{\leftskip 0pt\@startsection {section}{1}{\z@}%
	{-3.5ex \@plus -1ex \@minus -.2ex}%
	{2.3ex \@plus.2ex}%
	{\normalfont\Large\bfseries}}

\renewcommand\subsection{\leftskip 4ex\@startsection{subsection}{2}{\z@}%
	{-3.25ex\@plus -1ex \@minus -.2ex}%
	{1.5ex \@plus .2ex}%
	{\normalfont\large\bfseries}}

\renewcommand\subsubsection{\leftskip 14ex\@startsection{subsubsection}{3}{\z@}%
	{-3.25ex\@plus -1ex \@minus -.2ex}%
	{1.5ex \@plus .2ex}%
	{\normalfont\large\bfseries}}
\makeatother

%----------------------------------------------------------

\definecolor{titlepagecolor}{cmyk}{1,.60,0,.40}
\definecolor{namecolor}{cmyk}{1,.50,0,.10} 

\newcolumntype{C}{>{\centering\arraybackslash}X}

\title{ Economic History, Historical Analysis, and the "New History of Capitalism }
\author{Davide Davies-Biletta}
\date{\today}

\begin{document}
	\maketitle
	\setcounter{page}{1}
	\begin{spacing}{1.5}
		\bb
		The paper "Economic History, Historical Analysis, and the New History of Capitalism" by Eric Hilt is a critical survey of ten books which fall under a newly coined category referred to as \textit{History of Capitalism} \parencite{hilt2017economic}. History of Capitalism is a newly emerging subfield of history rather than economics. In this way, it differs from the field of Economic History in which most practitioners primarily consider themselves to be economists. Yet this is not the only way in which the two fields differ. Hilt's paper aims to further elucidate these distinct properties and shed light on the issues he fears face this innovative approach.
		\bb
		The emergence of the History of Capitalism represents a rejuvenated, and most-welcome interest in economic history by history scholars. Yet, while \cite{beckert2012history} defines the field simply as any economic research done by history scholars, Hilt claims that there is a distinctive and nuanced nature to this emerging discipline. Namely, rather than purely analysing historical behaviour, he considers that historians of capitalism place a particular emphasis on social criticism, providing a unique perspective.
		Hilt views this as one of the subdomain's greatest strengths. These critical perspectives distinguish their work from that of economic historians and make it relevant to the concerns of many popular readers. For this reason, the works of historians of capitalism have found popularity amongst a vastly wider audience than those of economic historians, which is mostly limited to academics. Furthermore, much of the research of economic historians focuses on questions originating in economic theory, which tend to be quite narrow. Meanwhile historians of capitalism present expansive narratives and explore questions that may not be amenable to the analytical tools of economists. Once again, this broader scope helps to attract a wider audience. 
		
		The new history of capitalism tends to emphasize the darker side of capitalism and its propensity for crises. It often neglects the positive aspects of capitalism, such as economic growth, innovation, and social mobility
		
		\bb
		However, in spite of its popularity, Hilt outlines serious epistemological concerns. He claims that in their pursuit of social criticism, historians of capitalism may sometimes lose focus on the need for rigorous historical analysis. The delicate balance between critical perspectives and methodological rigour requires constant attention to ensure the integrity of the research. Without this balance, and attention to the validity of their sources, historians of capitalism undermine their own work and take away from their effectiveness as social criticism. Importantly, Hilt urges that historians of capitalism often make no attempt to falsify their theories.  
		\bb
		Moreover, Hilt emphasises how there has been a failure for historians of capitalisms to engage with the economic history literature. Economic historians have produced sophisticated analyses of the issues of interest to historians of capitalism, even if they have approached them with a different set of tools. Ignoring the economic history literature has led historians of capitalism to make assertions that have been refuted conclusively and to get important elements of their arguments wrong. In some cases, historians of capitalism not only go against findings of
		
		\bb
		\textbf{Proposed solutions}\\
		Hilt underscores the importance of integrating institutional analysis into the study of the History of Capitalism. Despite acknowledging the unique strengths of the history field, the author suggests that historians of capitalism could benefit from incorporating analytical constructs and techniques commonly employed by economic historians, particularly those related to institutions. Notably, the discussion emphasizes the relative absence of institutional analysis within the literature on historians of capitalism. The author contends that such an incorporation would enhance the depth and insight of their works. Using the institution of slavery as an example, the text argues that understanding the implications of historical institutions, such as legal and economic frameworks, is essential for comprehending the long-term economic impact of historical events.
		\bb
		Hilt also advocates for the incorporation of counterfactuals in the study of History of Capitalism. Counterfactuals, a prominent tool in economic research, play a central role in disentangling causal relationships. While debates persist among history scholars regarding the validity of counterfactuals \parencite{evans2014altered}, Hilt underscores their significance in historical analysis. Furthermore, Hilt contends that books authored by historians of capitalism inherently make causal statements, inherently containing implicit counterfactuals. Despite this, he notes a deficiency in the explicit expression and thorough evaluation of these counterfactuals. A meticulous examination of these counterfactuals, Hilt argues, could have potentially yielded different, or perhaps more nuanced, conclusions. The absence of clear articulation and in-depth evaluation of these counterfactuals presents a noteworthy gap in the literature, one that Hilt suggests demands greater attention and scrutiny.
		
		
		\bb
		It is important to note that Eric Hilt, the author of this paper is trained as an economist. Checking his curriculum vitae we see that he completed his doctorate in Economics. This prompts valid questions about the legitimacy of Hilt's standpoint in challenging the sub-field of History of Capitalism as a whole.
		Upon scrutinizing Hilt's curriculum vitae, it becomes apparent that, according to Beckert's definition of History of Capitalism, contributions to the literature are typically confined to history scholars \parencite{beckert2012history}. This restriction poses potential challenges for Hilt. If this is the case, who is he to offer critique and solutions to things which practitioners may not even view as problems. He could be susceptible to accusations of engaging in "armchair quarterbacking." However, while Hilt's own definition is in itself vague, we have seen earlier why it might be better than the Beakert's. According to Hilt's definition, contributions to the History of Capitalism literature is not restricted solely to historians. This broader approach allows Hilt, as an economist, to provide valuable insights and critique.
		
		
		
		
		\bb
		\textbf{Conclusion}\\
		
		
		
		\pagebreak
		\printbibliography
		
	\end{spacing}
\end{document}