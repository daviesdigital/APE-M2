\documentclass[11pt]{article}

%use European style
\usepackage[a4paper,left=2cm,right=2cm,top=2cm,bottom=2cm]{geometry}

%few useful packages ------------------------------------------------------------------
\usepackage{setspace}
\let\Tiny=\tiny %remove annoying warnings
\usepackage[english]{babel}
\usepackage[latin1]{inputenc}
\usepackage{amsmath}
\usepackage{amssymb}
\usepackage{amsthm}
\usepackage{amsfonts}
\usepackage{colortbl}
\usepackage{xcolor}
\usepackage{eurosym}
\usepackage{enumitem}
\usepackage{chngpage}
\usepackage{fancyhdr}
\usepackage{fancyvrb}
\usepackage{float}
\usepackage{framed}
\usepackage{multirow}
\usepackage{graphicx}
\graphicspath{ {./images/} }
\usepackage{geometry}
\usepackage{lipsum}
\usepackage{tabularx}
\usepackage[linktocpage]{hyperref}

\usepackage{tikz}
\usetikzlibrary{positioning}

%define environment for code
\definecolor{orangepse}{RGB}{240,139,39}
\definecolor{redpse}{RGB}{222,6,61}
\newcommand{\rpse}[1]{\textcolor{redpse}{#1}}
\definecolor{dkgreen}{rgb}{0,0.6,0}
\definecolor{gray}{rgb}{0.5,0.5,0.5}
\definecolor{mauve}{rgb}{0.58,0,0.82}

\usepackage{listings}
\lstset{frame=tblr,
	language=R,
	aboveskip=5mm,
	belowskip=5mm,
	showstringspaces=false,
	columns=flexible,
	basicstyle={\small\ttfamily},
	numbers=none,
	numberstyle=\tiny\color{gray},
	keywordstyle=\color{blue},
	commentstyle=\color{dkgreen},
	stringstyle=\color{mauve},
	breaklines=true,
	breakatwhitespace=true,
	tabsize=3
}
%---------------------------------------------------------------------------------------
\usepackage[
backend=biber,
style=apa
]{biblatex}
\addbibresource{bibp12.bib}

% New Commands ----------------------------
\newcommand{\bb}{\bigbreak\noindent}

\makeatletter
\renewcommand\section{\leftskip 0pt\@startsection {section}{1}{\z@}%
	{-3.5ex \@plus -1ex \@minus -.2ex}%
	{2.3ex \@plus.2ex}%
	{\normalfont\Large\bfseries}}

\renewcommand\subsection{\leftskip 4ex\@startsection{subsection}{2}{\z@}%
	{-3.25ex\@plus -1ex \@minus -.2ex}%
	{1.5ex \@plus .2ex}%
	{\normalfont\large\bfseries}}

\renewcommand\subsubsection{\leftskip 14ex\@startsection{subsubsection}{3}{\z@}%
	{-3.25ex\@plus -1ex \@minus -.2ex}%
	{1.5ex \@plus .2ex}%
	{\normalfont\large\bfseries}}
\makeatother

%----------------------------------------------------------

\definecolor{titlepagecolor}{cmyk}{1,.60,0,.40}
\definecolor{namecolor}{cmyk}{1,.50,0,.10} 

\newcolumntype{C}{>{\centering\arraybackslash}X}

\title{What do unions do?}
\author{Davide Davies-Biletta}
\date{\today}

\begin{document}
	\maketitle
	\setcounter{page}{1}
	\begin{spacing}{1.7}
	
	\section{Review}
	Traditionally, unions have been associated with a reduction in aggregate productivity \parencite{meltzer1965labor,kuhn1998unions}. While even neoclassical theory can account for individual firms experiencing positive productivity boosts as a result of unionisation, at the aggregate level, unions are said to be socially undesirable. Opponents claim that unionised industries employ higher than optimal levels of capital, and are associated with a reduction in productivity growth. Union membership has fallen dramatically over the previous decades yet the shifting dynamics in the labour market, technological advancements, and globalization have prompted a reevaluation of traditional perspectives on unions. As the conversation continues, the future of collective action and unionisation remains a focal point in discussions about labour, productivity, and social equity.
	
	\bb
	In their chapter, and in this chapter specifically, Freeman and Medoff offer an alternative perspective of trade unionism in the United States of America \parencite{freeman1985unions}. This chapter builds upon their own research and the research of others authors offering similar perspectives, which were deemed controversial at the time. The authors outline a model which accounts for several distinct paths through which unionisation directly impacts firm productivity. The first path is the monopoly/monopolistic path through which firms respond to unionism by altering capital and other inputs per worker and improving the quality of labour until the contribution of the last unit of labour just equals the union wage rate. This model can also account for unionisation leading to lower productivity, yet the authors argue this is actually unlikely to happen in competitive markets as firms who pay more for less productivity will soon go out of business. Productivity gains through this path however are considered socially undesirable for the reasons discussed in the previous paragraph. 
	
	The second path, on the other hand, is the voice/response path. This was first outlined in a book entitled Exit, Voice, and Loyalty \parencite{hirschman1972exit}.  The ``voice" refers to the process of direct
	communication designed to bring actual and desired conditions closer together. In modern industrial economies, particularly in large enterprises, a trade union is the vehicle for collective voice, providing workers as a group with a means of communicating with management \parencite{freeman1979two}. Therefore, through this path, productivity gains  come from improved efficiency associated discussions between the worker and the employer. Instead of simply quitting the worker can make his grievences heard and the employer can choose to remedy them. This is the response part of the ``voice/response path''. The subsequent lower staff turn over results in lower costs to training and recruitment and therefore increased efficiency. Hence, productivity gains from this path can be considered socially desirable. The inclusion of this novel pathway in the model therefore allows the authors to directly account for positive impacts of unions on productivity.
	 
	%To see if unionism is positively or negatively related to the growth of productivity, as opposed to the level of productivity, we have analyzed the impact Of the proportion of workers unionized on the rate of growth of value added or value of shipments in three data sets. The results of our analysis, which are summarized in table 11?2, suggest that while unionized industries have, indeed, had somewhat slower growth of productivity than nonunion sectors, the observed relation is too weak statistically to support the claim that unionism reduces dynamic efficiency. Some unionized industries have rapid productivity growth while others have less rapid growth. Because unionized sectors tend to be  older" industries, one expects some negative relation between productivity and unionism because of the life cycle of industries (a growing new industry typically enjoys more rapid productivity growth than an older established industry), even if unionism did nothing harmful to the rate Of industrial progress. In sum, current empirical evidence offers little support for the assertion that unionization is associated with lower (or higher) productivity advance.
	
	\bb\bb
	Furthermore, the addition of this voice/response model brings with it previously overlooked factors such as the state of labour-management relationships. The authors provide evidence to suggest that the impact of unionism on productivity depends not only on what unions and management do separately but on their relationship with one another. Evidence suggests that where this relationship is good, productivity is higher than in plants where it is bad. Emphasizing the collaborative nature of the voice/response dynamic, the success of this approach hinges on the capacity for open and constructive communication within the workplace. In modern industrial settings, trade unions serve as the conduit for collective voice, allowing workers to articulate concerns and preferences directly to management \parencite{freeman1979two}. However, the efficacy of this mechanism doesn't solely rest on the ability of workers to express their grievances; it equally demands a receptive and responsive stance from management. For the voice/response path to yield positive outcomes, it necessitates not only a workforce comfortable and capable of identifying issues but also a management willing to listen, comprehend, and take affirmative actions to address those concerns. The success of this approach lies in the reciprocal nature of the interaction, where the ``response" involves managerial acknowledgment and resolution of issues raised by the workforce. Therefore, the state of labor-management relationships plays a pivotal role in determining the success of this model.
	
	\section{Critical Analysis}
	The results outlined in this paper can be said to be mostly correlational at best. There is no serious attempt by the authors at unearthing a causal effect of unionisation on productivity. Randomly distributing unions across industries would be impossible or at least impractical, and potentially unethical, however both a difference-in-difference or regression discontinuity design might be more appropriate. The use of these quasi-experimental methods would prove useful in establishing a more robust causal relationship and may go some way to furthering the discussion. \cite{freeman1980unionism} attempts to include a econometric component yet they fail to include a time series analysis. Another paper, from the same year, by \cite{clark1980impact} is favoured by the authors due to the inclusion of this time series analysis. However, Clark's work has shown a both positive and negative effects of unions on productivity and may weaken the central argument made by the authors in their chapter, and emphasizing the importance of critically assessing and reconciling divergent findings within the existing literature.
	 
	\bb
	This chapter serves as an introduction to the topic of unionism's varied effect on productivity. The authors' motivation, distinct from that of a stand-alone academic article, sets the stage for a broader examination of the topic. However, the authors rely on a small set of papers to emphasise their argument. The adoption of a more meta-analytical approach, such as the work of \cite{doucouliagos2003unions}, could offer a more complete picture of existing studies. A meta-analysis provides the opportunity to weigh the collective evidence systematically, identifying patterns and disparities across various contexts. This approach would lead to a more convincing and complete argument by overcoming the limitations of the individual studies mentioned in their current chapter. 
	
	\bb
	Furthermore, the paper neglects to address critical issues related to data validity in measuring productivity. Productivity is a difficult concept to define let alone measure. Previous papers referenced in this text measure labour productivity differently, making them hard to compare. Some of these papers measure it in dollar units (value added by the firm or the value of shipments) per worker, and others measuring it in physical units (tons, square feet) per worker. The dollar measures of output (price times quantity) have the advantage of including the full spectrum of goods produced by a firm, valued at their market prices. However, unless the prices charged by union and non-union firms are the same, any finding of higher value added (shipments) per worker in the organized establishments could reflect not the higher physical output per worker but rather a higher price per unit of output. This means that theoretical findings are only really applicable to the truly competitive markets. Meanwhile, physical measures of output  have there own issues. While they do alleviate the problem of confusing price differences for output differences, they are limited to the few distinct goods that can be so measured. Most modern firms produce a wide variety of products with too many dimensions to be captured by a single physical measure. The authors acknowledge the complexity but fall short of providing a comprehensive resolution to this measurement issue for future research. 
	
	
	\section{Conclusion}
	In conclusion, Freeman and Medoff's work was one of the first to challenge the traditional notions of unions as solely detrimental to aggregate productivity. Their model, encompassing both the monopoly/monopolistic path and the voice/response path, broadens our understanding by accounting for diverse mechanisms through which unions influence firm productivity. The identification of the collaborative nature of the voice/response dynamic underscores the importance of positive labour-management relationships, where open communication and responsive actions contribute to enhanced productivity. However, a critical analysis reveals the correlational nature of the presented results, urging a need for more robust quasi-experimental designs to establish causation. Furthermore, the authors' reliance on a limited set of papers and the absence of a comprehensive meta-analytical approach leave room for a more exhaustive examination of the literature. Addressing issues related to data validity in measuring productivity remains a critical aspect that the paper does not thoroughly explore.

	Overall, Freeman and Medoff's work provides a valuable contribution to the literature and challenging the prevalent unfavourable attitude towards unions at the time as outlined in \cite{freeman1979two}. The number of citations alone provides an insight to the impact this work has had subsequently. In spite of the methodological limitations, the authors provided a foundation on which future research was to be based. 

	
	\end{spacing}
\end{document}
